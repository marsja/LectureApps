\documentclass[]{article}
\usepackage{pdflscape}
\usepackage{longtable}
\usepackage{booktabs}
\usepackage{everypage}

\newcommand{\Lpagenumber}{\ifdim\textwidth=\linewidth\else\bgroup
  \dimendef\margin=0 %use \margin instead of \dimen0
  \ifodd\value{page}\margin=\oddsidemargin
  \else\margin=\evensidemargin
  \fi
  \raisebox{\dimexpr -\topmargin-\headheight-\headsep-0.5\linewidth}[0pt][0pt]{%
    \rlap{\hspace{\dimexpr \margin+\textheight+\footskip}%
    \llap{\rotatebox{90}{ CV  - Erik Marsja 
- \thepage\hspace{1pt}(\pageref{LastPage})}}}}%
\egroup\fi}
\AddEverypageHook{\Lpagenumber}%

\usepackage{adjustbox}
\usepackage{titling}
\usepackage{caption}
\usepackage{tabularx}
\usepackage{tikz}
\usepackage{xcolor}
\usepackage{fancyhdr}
\usepackage{lastpage}
\usepackage{titlesec}
\usepackage[scaled=.80]{helvet}% Helvetica, served as a model for arial

\pagestyle{fancy}
\fancyhf{}
\renewcommand{\headrulewidth}{0pt}
\renewcommand{\footrulewidth}{0.0pt}
\fancyfoot[CO,CE]{   Erik Marsja -  \thepage\hspace{1pt}(\pageref{LastPage})}
\fancypagestyle{plain}{\pagestyle{fancy}}

\usepackage[margin=1in]{geometry}

% Font awesome
\usepackage{fontawesome}

\usepackage[T1]{fontenc}
\usepackage[utf8]{inputenc}

% For tables


% Tightlist

\providecommand{\tightlist}{%
  \setlength{\itemsep}{0pt}\setlength{\parskip}{0pt}}

% URLS
\usepackage[hidelinks]{hyperref}
\usepackage{breakurl}
\usepackage{float} % here for H placement parameter

% Appendix header style

\fancypagestyle{style2}{
\fancyhf{}
\fancyhead[C]{Appendix 1}
}


% For Changing the margins (Education)

\def\changemargin#1#2{\list{}{\rightmargin#2\leftmargin#1}\item[]}
\let\endchangemargin=\endlist 

%For displaying the table in landscape format
\usepackage[absolute]{textpos}

\fancypagestyle{lscape}{% 
\fancyhf{} % clear all header and footer fields 
\fancyfoot[LE]{%
\begin{textblock}{20}(1,5){\rotatebox{90}{\leftmark}}\end{textblock}
\begin{textblock}{1}(13,10.5){\rotatebox{90}{\thepage}}\end{textblock}}
\fancyfoot[LO] {%
\begin{textblock}{1}(13,10.5){\rotatebox{90}{\thepage}}\end{textblock}
\begin{textblock}{20}(1,13.25){\rotatebox{90}{\rightmark}}\end{textblock}}
\renewcommand{\headrulewidth}{0pt} 
\renewcommand{\footrulewidth}{0pt}}

\setlength{\TPHorizModule}{1cm}
\setlength{\TPVertModule}{1cm}

% Appendix
\fancypagestyle{style2}{
\fancyhf{}
\fancyhead[C]{Appendix 1}
\renewcommand{\headrulewidth}{1pt}
}



\newcommand\secbar {
    \tikz[baseline, trim left=3.2cm] 
    {
        \fill [white] (3cm,0.1ex) rectangle +(0.2cm,1.1ex);
        \draw [gray!95, fill=gray!80] (0cm,0.1ex) rectangle (3cm,1.1ex);        
    }
}
\newcommand\subsecbar {
    \tikz[baseline, trim left=0.15cm] 
    {
        \fill [white] (2cm,0.1ex) rectangle +(0.2cm,1.1ex);
        \fill [blue!40] (0cm,0.1ex) rectangle (2cm,1.1ex);      
    }
}

\newcommand\subsubsecbar {
    \tikz[baseline, trim left=0.15cm] 
    {
        \fill [white] (1cm,0.1ex) rectangle +(0.2cm,1.1ex);
        \fill [blue!40] (0cm,0.1ex) rectangle (2cm,1.1ex);      
    }
}

\titleformat{\section}{\large}{}{0cm}{\secbar}
\titleformat{\subsection}{\large}{}{0cm}{\normalfont\sffamily\Large\bfseries\subsecbar}
\titleformat{\subsubsection}{}{}{0cm}{\normalfont\sffamily\large\bfseries}

% No first line paragraph indent
\usepackage{parskip}
\usepackage{enumitem}

\setlength{\parskip}{0cm}


\titlespacing\section{0pt}{12pt plus 4pt minus 2pt}{1pt plus 1pt minus 2pt}
\titlespacing\subsection{0pt}{12pt plus 4pt minus 2pt}{1pt plus 1pt minus 2pt}
\titlespacing\subsubsection{0pt}{12pt plus 4pt minus 2pt}{4pt plus 2pt minus 2pt}

\newcolumntype{Y}{>{\centering\arraybackslash}X}

\begin{document}

\centerline{\huge \textbf{Erik
Marsja} | \textcolor{darkgray}{Avsiktsförklaring bitr. lektorat}}

\vspace{1mm}

\hrule

\begin{table}[h]
\centering
\begin{tabularx}{\textwidth}{@{}lYl@{}}
\textbf{Home Address}: & &  \textbf{Personal Identity Number:} 
\\Tvistevägen 26, SE-907 36 Umeå, Sweden & &  19810526 
\\\\

 \faPhone \hspace{1 mm}  46703633662  \hspace{1 mm}  &  & \faEnvelopeO \hspace{1 mm} \href{mailto:}{\tt \href{mailto:erik.marsja@liu.se}{\nolinkurl{erik.marsja@liu.se}}} \hspace{1 mm}  \\
 \faGlobe \hspace{1 mm} \href{http://www.marsja.se}{\tt www.marsja.se}   &  & \faGithub \hspace{1 mm} \href{http://github.com/marsja}{\tt marsja} \hspace{1 mm}  \\
 \multicolumn{3}{c}{}
 \\\hline
\end{tabularx}
\end{table}

Till sakkunniga, anställningskommitté, och övriga berörda
representanter,

\hfill\break
Med denna avsiktsförklaring vill jag deklarera mina avsikter för
tjänsten som biträdande lektor vid Avdelningen för Funktionsnedsättning
och Samhälle, Institutionen för Beteendevetenskap och Lärande,
Linköpings Universitet.

\hypertarget{forskning}{%
\subsection{Forskning}\label{forskning}}

I min forskning har jag fram till idag främst använt mig av olika
kvantitativa ansatser såsom till exempel experiment och analys av data
från databaser. De flesta projekt jag har varit involverad i faller inom
ramen för kognitiv hörselvetenskap men jag har även studerat andra
sensoriska modaliteter. Under min tjänst som postdoktor vid avdelningen
för funktionsnedsätting och samhälle har min forskning främst handlat om
analys av data från N200-databasen (n = ca. 500) med fokus på kognitiva
mekanismer för taluppfattning i bakgrundsbrus. Vidare har jag analyserat
data från experiment med samma fokus. Forskningen har inneburit allt
ifrån enkel statistisk analys till avancerad statistisk analys (exv.
linear mixed-effects modeling och strukturell ekvationsmodellering).
Dessa studier har undersökt eventuella skillnader och likheter mellan
olika grupper (exv. äldre och unga, personer med eller utan
hörselnedsättning). Med syftet att undersöka hur kognitiva mekanismer
kan vara olika relaterade till taluppfattning i brus för olika
åldersgrupper, eller är beroende av hörselstatus.

\hfill\break
Den forskning jag avser att bedriva vid Avdelningen för
Funktionsnedsättning och Samhälle bygger i det stora hela på ovan nämnda
tidigare forskning och den typ av forskning jag bedrev under min tid som
forskarstuderande. Min framtida forskning kan grovt delas in i tre
huvudområden kognitiv hörselvetenskap, psykologi, och transport. Den
gemensamma nämnaren för den forskning jag avser att bedriva är inom
området funktionsnedsättning och samhälle utifrån bland annat den
bio-psykosociala modellen. Trots att forskningen jag bedrivit, och
planerar att bedriva, främst har en kvantitativ prägel så anser jag att
forskningsfrågan avgör den metodologiska ansatsen. Det vill säga, jag
ämnar bedriva min forskning med både kvantitativ och kvalitativ metod,
såväl som blandningen av dessa ansatser(mixed-methods) beroende på
vilken som är mest lämplig. Utifrån bredden inom forskningsfältet anser
jag att varje metod har sin givna plats.

\hfill\break
Eftersom N200-projektet är en longitudinell studie avser jag att följa
upp en del av den forskning jag redan publicerat. Ett exempel är att jag
ämnar studera hur åldrande påverkar hörsel och kognition. Detta skulle,
till exempel, kunna innefatta användandet av avancerade statistiska
metoder som latent growth curve eller latent change score modeling.
Vidare avser jag även att utveckla mina egna forskningsområden inom
kognitiv hörselvetenskap. Till exempel, vill jag undersöka hur mycket av
de kognitiva mekanismerna bakomliggande taluppfattning som har med
språkliga färdigheter att göra.

\hfill\break
Jag avser att utöka den forskning som rört vid psykosociala aspekter av
hörselnedsättning (subjektiva mått av vardagliga hörselproblem). Till
exempel avser jag att undersöka sambandet mellan psykosocial hälsa och
läs- och skrivförmåga hos elever med hörselnedsättning i mellan- och
lågstadiet. Ett annat exempel på forskning jag avser att bedriva med en
psykosocial prägel är den med inriktning på hörselnedsättning och
digitaliseringen av samhället i synnerhet arbetslivet. Preliminära
resultat från min egen forskning antyder att digitaliseringen (exv.
videomöten i större utsträckning) av arbetslivet kan ha positiva
effekter, både fysiskt och psykiskt. Eftersom dessa resultat är från en
intervjustudie avser jag att följa upp dem med en kvantitativ ansats
tillsammans med kollegor inom olika områden, bland annat
funktionsnedsättning och samhälle. Min förhoppning är att utveckla
forskningen så att den innefattar fler kognitiva och psykosociala
aspekter (exv. stress och konsekvenser som ohälsa och sjukskrivning).
Jag vill även undersöka hur hjälpmedel kan underlätta i videomöten, till
exempel undertexter. Andra funktionsnedsättningar kan även vara av
intresse inom detta forskningsområde.

\hfill\break
Under min tid som forskare vid Statens Nationella Väg- och
Transportforskningsinstitut (VTI) har jag utökat mitt kontaktnät med
forskare med fokus på funktionsnedsättningar i transportsektorn.
Tillsammans med dessa avser jag att undersöka vilka barriärer det finns
för äldre med åldersrelaterade funktionsnedsättningar (exv. syn- och
hörselnedsättningar) i kollektivtrafiken. Denna typ av forskning, dvs.
transportforskning med fokus på personer med funktionsnedsättning, avser
jag att fortsätta med inom ramen för Linköpings styrkeområde Transport.
Här avser jag främst att min forskning ska bidra direkt till samhället
genom att stimulera till mobilitet, delaktighet, och rättvisa för att
till exempel främja psykisk hälsa hos individer med
funktionsnedsättning.

\hfill\break
Jag avser vidare att samarbeta med forskare både nationellt och
internationellt inom området funktionsnedsättning och samhälle men även
med forskare inom områdena psykologi och pedagogik. Ett exempel är det
påbörjade samarbetet med forskare vid Avdelningen för Psykologi vid
Linköpings Universitet. Detta projekt är ett exempel på mina
internationella samarbeten. Det finns även påbörjade samarbeten
nationellt inom ramen för funktionsnedsättning och samhälle där vi avser
att studera ADHD baserat på studier från min avhandling.

\hfill\break
Summa summarum, jag avser att vara under ständig utveckling inom området
för funktionsnedsättning och samhälle. Detta innebär att; även om jag
avser att fokusera min forskning på hörselnedsättning, kognitiva
mekanismer, och funktionsnedsättningens samhälleliga konsekvenser, så
strävar jag att även undersöka andra funktionsnedsättningar. Jag avser
att utöka min kunskap både i kvantitativ och kvalitativ metod. Jag
avser, för att nämna ett exempel, att lära mig avancerade statistiska
metoder inom fältet för artificiell intelligens (exv. deep learning,
neurala nätverk) för att kunna tillämpa detta på olika typer av data
(exv. beteende data från video).

\hypertarget{undervisning}{%
\subsection{Undervisning}\label{undervisning}}

När det kommer till undervisning har jag för avsikt att bedriva och
vidareutveckla avdelningens kurser inom ämnesområdet
funktionsnedsättning och samhälle. Vidare avser jag att vara en aktiv
del av lärarkollegiet. Jag avser att bidra till arbetet med
kvalitetsgranskning och omarbetning av kurser i samarbete med
studierektor och övriga lärare. Utöver avdelningens egna kurser avser
jag att kunna bidra med undervisning kopplat till min forskningsexpertis
vid efterfrågan. Detta gäller både andra utbildningar vid institutionen
(exv. Psykologprogrammet) men även vid andra institutioner och
fakulteter vid Linköpings Universitet (exv. kognitionsvetenskapliga
programmen vid Institutionen för Datavetenskap).

\hfill\break
Som jag tidigare antytt, är det viktigt för mig att utvecklas och det
gäller givetvis även min undervisning. För att utvecklas som lärare,
utöka mina kunskaper i högskolepedagogik och kunna bedriva undervisning
på forskarutbildningsnivå har jag för avsikt att läsa fler kurser i
universitetspedagogik. Till exempel kommer jag att läsa Linköpings
Universitets påbyggnadskurs i högskolepedagogik och kurser gällande
handledning på forskarutbildningsnivå. Att utvecklas pedagogiskt innebär
givetvis även att jag avser att delta i Lärarluncher, lärardagar, och
andra forum för att utvecklas genom att exempelvis dela andras
erfarenheter och att få råd. Slutligen, eftersom jag tidigare inte haft
möjlighet att leda en kurs som kursansvarig så avser jag här att få
driva och utveckla en kurs eller ett delmoment inom funktionsnedsättning
och samhälle.

\hypertarget{samverkan}{%
\subsection{Samverkan}\label{samverkan}}

Jag avser även att utöka min samverkan med samhället genom att till
exempel fortsätta skriva om forskning, och metod, på min personliga
blogg, att inom ramen som expert i de Hörselskadades riksförbund bidra
med information och kunskap från forskningen vid Avdelningen för
Funktionsnedsättning och Samhälle. Vidare, ämnar jag att kommunicera ut
forskning från avdelningen, framförallt min egen och den forskargrupp
jag tillhör, till berörda aktörer i samhället (exv. hörselkliniker,
hjälpmedelcentrum).

\hypertarget{arbetsmiljuxf6}{%
\subsection{Arbetsmiljö}\label{arbetsmiljuxf6}}

Vidare är min avsikt att delta aktivt i möten och aktiviteter på alla
nivåer på Linköpings Universitet. Det vill säga, möten och aktiviteter
anordnade på avdelnings-, institutions- samt fakultetsnivå. Jag kommer
att bidra till avdelningens utveckling och arbete i form av aktivt
deltagande vid avdelningsmöten, lärarmöten och avdelningens
seminarieserier samt ta ett ledaransvar när detta behövs. Mitt
deltagande innebär att jag kommer verka för ett trevligt, respektfullt
och inkluderande klimat på avdelningen samt inom de arbetsgrupper där
jag är verksam.

\hfill\break
Utifrån mina publikationer jag hittills gjort under min tid på
Avdelningen för Funktionsnedsättning och Samhälle har jag ett mål
gällande min meritering. Mer konkret avser jag att meritera mig så att
jag kan ansöka om docentur inom två år. Ett annat mål är att jag även
avser att handleda studenter från kandidat- till doktorandnivå.

\hfill\break
Avslutningsvis, avser jag att bedriva min forskning inom ramen för Open
Science genom att pre-registrera datainsamlingar när det är lämpligt
(exv. experiment, eller planerade analyser av registerdata),
tillgängliggöra data (när möjligt), RMarkdown-filer och analysskript för
att öka reproducerbarheten av mina forskningsresultat. Min kunskap och
min strävan för att verka för öppen vetenskap avser jag sprida till
kollegor vid avdelningen och eventuella studenter, framför allt men inte
uteslutande, på forskarnivå.

\hfill\break
\footnotesize Detta dokument är skapat med RMarkdown och alla filer för
att skapa dokumentet, såväl som alla mina andra ansökningshandlingarm
finns tilllängliga på GitHub:
\url{https://github.com/marsja/LectureApps}

\end{document}