\documentclass[]{article}
\usepackage{pdflscape}
\usepackage{longtable}
\usepackage{booktabs}
\usepackage{everypage}

\newcommand{\Lpagenumber}{\ifdim\textwidth=\linewidth\else\bgroup
  \dimendef\margin=0 %use \margin instead of \dimen0
  \ifodd\value{page}\margin=\oddsidemargin
  \else\margin=\evensidemargin
  \fi
  \raisebox{\dimexpr -\topmargin-\headheight-\headsep-0.5\linewidth}[0pt][0pt]{%
    \rlap{\hspace{\dimexpr \margin+\textheight+\footskip}%
    \llap{\rotatebox{90}{ CV  - Erik Marsja 
- \thepage\hspace{1pt}(\pageref{LastPage})}}}}%
\egroup\fi}
\AddEverypageHook{\Lpagenumber}%

\usepackage{adjustbox}
\usepackage{titling}
\usepackage{caption}
\usepackage{tabularx}
\usepackage{tikz}
\usepackage{xcolor}
\usepackage{fancyhdr}
\usepackage{lastpage}
\usepackage{titlesec}
\usepackage[scaled=.90]{helvet}% Helvetica, served as a model for arial

\pagestyle{fancy}
\fancyhf{}
\renewcommand{\headrulewidth}{0pt}
\renewcommand{\footrulewidth}{0.0pt}
\fancyfoot[CO,CE]{   Redog\"{o}relse f\"{o}r vetenskaplig verksamhet \\  Erik Marsja -  \thepage\hspace{1pt}(\pageref{LastPage})}
\fancypagestyle{plain}{\pagestyle{fancy}}

\usepackage[margin=1.2in]{geometry}

% Font awesome
\usepackage{fontawesome}

\usepackage[T1]{fontenc}
\usepackage[utf8]{inputenc}

% For tables


% Tightlist

\providecommand{\tightlist}{%
  \setlength{\itemsep}{0pt}\setlength{\parskip}{0pt}}

% URLS
\usepackage[hidelinks]{hyperref}
\usepackage{breakurl}
\usepackage{float} % here for H placement parameter

% Appendix header style

\fancypagestyle{style2}{
\fancyhf{}
\fancyhead[C]{Appendix 1}
}


% For Changing the margins (Education)

\def\changemargin#1#2{\list{}{\rightmargin#2\leftmargin#1}\item[]}
\let\endchangemargin=\endlist 

%For displaying the table in landscape format
\usepackage[absolute]{textpos}

\fancypagestyle{lscape}{% 
\fancyhf{} % clear all header and footer fields 
\fancyfoot[LE]{%
\begin{textblock}{20}(1,5){\rotatebox{90}{\leftmark}}\end{textblock}
\begin{textblock}{1}(13,10.5){\rotatebox{90}{\thepage}}\end{textblock}}
\fancyfoot[LO] {%
\begin{textblock}{1}(13,10.5){\rotatebox{90}{\thepage}}\end{textblock}
\begin{textblock}{20}(1,13.25){\rotatebox{90}{\rightmark}}\end{textblock}}
\renewcommand{\headrulewidth}{0pt} 
\renewcommand{\footrulewidth}{0pt}}

\setlength{\TPHorizModule}{1cm}
\setlength{\TPVertModule}{1cm}

% Appendix
\fancypagestyle{style2}{
\fancyhf{}
\fancyhead[C]{Appendix 1}
\renewcommand{\headrulewidth}{1pt}
}



\newcommand\secbar {
    \tikz[baseline, trim left=3.2cm] 
    {
        \fill [white] (3cm,0.1ex) rectangle +(0.2cm,1.1ex);
        \draw [gray!95, fill=gray!80] (0cm,0.1ex) rectangle (3cm,1.1ex);        
    }
}
\newcommand\subsecbar {
    \tikz[baseline, trim left=0.15cm] 
    {
        \fill [white] (2cm,0.1ex) rectangle +(0.2cm,1.1ex);
        \fill [blue!40] (0cm,0.1ex) rectangle (2cm,1.1ex);      
    }
}

\newcommand\subsubsecbar {
    \tikz[baseline, trim left=0.15cm] 
    {
        \fill [white] (1cm,0.1ex) rectangle +(0.2cm,1.1ex);
        \fill [blue!40] (0cm,0.1ex) rectangle (2cm,1.1ex);      
    }
}

\titleformat{\section}{\large}{}{0cm}{\secbar}
\titleformat{\subsection}{\large}{}{0cm}{\normalfont\sffamily\Large\bfseries\subsecbar}
\titleformat{\subsubsection}{}{}{0cm}{\normalfont\sffamily\large\bfseries}

% No first line paragraph indent
\usepackage{parskip}
\usepackage{enumitem}


\titlespacing\section{0pt}{12pt plus 4pt minus 2pt}{4pt plus 2pt minus 2pt}
\titlespacing\subsection{0pt}{12pt plus 4pt minus 2pt}{4pt plus 2pt minus 2pt}
\titlespacing\subsubsection{0pt}{12pt plus 4pt minus 2pt}{4pt plus 2pt minus 2pt}

\newcolumntype{Y}{>{\centering\arraybackslash}X}

\begin{document}

\centerline{\huge \textbf{Erik Marsja} | \textcolor{darkgray}{Current \& Planned Scientific Projects}}

\vspace{2 mm}

\hrule

\begin{table}[h]
\centering
\begin{tabularx}{\textwidth}{@{}lYl@{}}
\textbf{Home Address}: & &  \textbf{Personal Identity Number:} 
\\Tvistevägen 26, SE-907 36 Umeå, Sweden & &  19810526 
\\\\

 \faPhone \hspace{1 mm}  0046703633662  \hspace{1 mm}  &  & \faEnvelopeO \hspace{1 mm} \href{mailto:}{\tt \href{mailto:erik@marsja.se}{\nolinkurl{erik@marsja.se}}} \hspace{1 mm}  \\
 \faGlobe \hspace{1 mm} \href{http://www.marsja.se}{\tt www.marsja.se}   &  & \faGithub \hspace{1 mm} \href{http://github.com/marsja}{\tt marsja} \hspace{1 mm}  \\
 \multicolumn{3}{c}{\emph{Languages: }Swedish, English}
 \\\hline
\end{tabularx}
\end{table}

\hypertarget{previous-and-current-research}{%
\subsection{Previous and Current
Research}\label{previous-and-current-research}}

In my doctoral thesis (Marsja, 2017), I investigated how irrelevant
unexpected auditory and tactile stimuli have an impact on visual
processing. In the first two studies of my dissertation, I used simple
visual categorization tasks and these two studies identified a knowledge
gap which leads to the third study: How do sudden changes in irrelevant
auditory, tactile, or bimodal (both tactile and auditory) sequences
affect short-term memory processes (Marsja, Marsh, Hansson, \& Neely,
2020)?

The results showed that distraction of unexpected tactile stimuli is
similar to distraction of unexpected auditory stimuli and that a
possible difference is that the effect of unexpected tactile stimuli
disappears over time (Marsja, Neely, K-Ljungberg, Under Review).
Furthermore, unexpected sounds presented among repeated vibrations only
capture attention if the vibration is not presented simultaneously as
the sound. Omitting a repeated vibration is enough to capture attention
(Marsja, Neely, K-Ljungberg, 2018). Regarding short-term memory
processing, the results showed that a change in spatial location of an
irrelevant sequence only disrupts short-term memory processes when the
irrelevant sequence consists of both sound and vibration (Marsja, Marsh,
Hansson, \& Neely, 2019)

Together with international and national short-term memory researchers,
I performed a series of 3 experiments with the aim of investigating what
disturbs visual short-term memory (Marsh, Vachon, Sörqvist, Marsja,
Röer, \& K-Ljungberg, Under Revision). Specifically, is the memory for
visual sequences disturbed when an irrelevant sequence of vibrations is
presented simultaneously? In this study, we used vibrotactile sequences
with a constant change (the sequence ``jumps'' between the two hands,
left-right-left-right-left-right; * changing-state * sequence), and
\emph{steady-state} sequences (when all vibrations in the sequence are
presented to both hands).

We found that visual short-term memory performance is more interfered by
a changing-state vibrotactile sequence compared to a steady-state
tactile sequence. The effect of a changing-state vibrotactile sequence
is, moreover, similar to that of the changing-state sequence consisting
of sound (Experiment 1); the interference between vibrotactile stimuli
and short-term memory seems to affect the recall of the order of objects
rather than article identity (Experiment 2), and the predictability of
vibrotactile stimuli does not appear to modulate the extent of the
effect (Experiment 3).

I have also studied the interaction between the auditory, visual, and
tactile modalities from a more applied perspective. In a study, we aimed
to determine how effective a tactile warning is during increasing levels
of mental workload in a primary task. In this study, we used three
simulated flight task conditions in which we varied the mental workload
while we presented an ``on-thigh'' vibrotactile warning to human
subjects. Generally, we found a decrement in overall warning response
performance when task workload increased, but this tendency faded and
plateaued as the level of task workload progressed. This pattern
indicates that vibrotactile warning signals may offer a plausible mode
for conveying information during increasing levels of primary task
workload (Rosa, Marsja, \& K-Ljungberg, 2020). Moreover, it was found
that increasing tasks in an interface leads to increased mental
workload.

More recently, I have extended my research into the field of cognitive
hearing science and disability research. Together with national
researchers, this research has focused on the relationship between
cognitive functioning, age, and speech in noise as well as subjective
aspects of speech in noise in relation to speech in noise performance.
In one study, we have used multivariate statistical methods (i.e.,
factor analysis and multigroup structural equation modeling) to examine
how cognitive functioning and age affect speech in noise performance in
individuals with, and without, hearing impairment. We found that
cognitive functioning and age have a similar impact whether you have a
hearing impairment or not (Marsja, Stenbäck, Moradi, Danielsson, \&
Rönnberg, Under Revision). Similarly, we have found that high working
memory capacity is related to lower scores of self-rated listening
effort for informational maskers, as well as better performance in
speech recognition in noise (Stenbäck, Marsja, Hällgren, Lyxell, \&
Larsby, Accepted). In another study, in which we used both self-reported
hearing measures (i.e., a questionnaire) and performance-based
behavioral measures (Stenbäck, Marsja, Ellis, \& Rönnberg, Under
Revision). We found no correlation between self-reported and behavioral
measures of speech in noise for individuals with hearing impairment.
However, this correlation was significant for individuals without
hearing impairment. To examine this further, we have also applied
machine learning techniques (e.g., random forest regression) to examine
the most important self-reported variables for speech in noise
(Stenbäck, Marsja, Danielsson, \& Rönnberg, In Preparation). Finally, we
are also examining how logical inference (both auditory and visually)
might be related to speech in noise recognition (Stenbäck, Marsja,
Danielsson, \& Rönnberg, In Preparation).

\hypertarget{planned-research-zoom-fatigue}{%
\subsection{Planned Research: ``Zoom
Fatigue''}\label{planned-research-zoom-fatigue}}

My recent research, in cognitive hearing science, has lead me to the
knowledge that hearing loss affects the ability to communicate and
establish and maintain social relationships. For people with hearing
loss, listening can be very effortful which can lead to a person
stopping using a hearing aid or choose not to participate in social
events. Communication has changed radically due to COVID-19. Most
measures, including working from home and the use of video calls, were
intended to be provisional. However, the digitalization of working life
(e.g meetings, conferences) is here to stay. Little is known about the
experiences and consequences of digitization in individuals with, or
without, hearing loss. Video calls can be affected by degraded sound
quality and asynchronous audio and video and create an unfavorable and
demanding listening situation for people with hearing loss. It is
therefore both theoretically important and relevant for society to study
how individuals with hearing impairment experience digitalization (i.e.,
video calls) and how it affects participation, both socially and
professionally. I, therefore, plan to investigate how digitization,
video calls in particular, of working life has affected people with
hearing loss. Specifically, I aim to investigate whether listening
becomes more effortful for people with hearing loss and whether this
harms the individual's participation in working life. In this project, I
plan two studies, one more focused on the past (i.e., during the covid
pandemic) that will use self-reported measures. This study aims to get a
rich understanding of how individuals with hearing loss perceive their
working life when it is digitalized. Here I plan to use data-driven
(e.g., machine learning methods). The second study, on the other hand,
is planned to use experiments in which participants will communicate
through video call software under adverse conditions (e.g., audiovisual
lag). This study will use the information from the first study to narrow
down the experimental conditions. In this project, I plan to collaborate
across disciplines and, in the long run, hope to guide designers of
interfaces (e.g., of video conferencing software). By increasing the
knowledge on what affects individuals with hearing loss, new algorithms
for e.g., audio compression can be developed.

In addition to the specific project on ``Zoom fatigue'', I also plan on
working on available large databases with my previous collaborators
(e.g., Dr.~Stenbäck and Prof.~Danielsson). For example, I plan to apply
machine learning algorithms to the n200 database to entangle e.g.~which
clinical tools are most efficient in diagnosing hearing loss (e.g., see
Dwyer, Falkai, \& Koutsouleris, 2018 for an overview of machine learning
methods in clinical psychology).

\hypertarget{references}{%
\subsubsection{References}\label{references}}

Dwyer, D. B., Falkai, P., \& Koutsouleris, N. (2018). Machine Learning
Approaches for Clinical Psychology and Psychiatry. Ssrn.
\url{https://doi.org/10.1146/annurev-clinpsy-032816-045037}

Marsh, J. E., Vachon, F., Sörqvist, P., Marsja, E., Röer, J. P., and
Ljungberg, J. K. (Manuscript in Preparation). Irrelevant vibro-tactile
stimuli produce a changing-state effect: Implications for theories of
interference in short-term memory.

Marsja, E., Neely, G., and Ljungberg, J. K. (2018). Investigating
Deviance Distraction and the Impact of the Modality of the To-Be-Ignored
Stimuli. Exp. Psychol. 65. \url{doi:10.1027/1618-3169/a000390}.

Marsja, E. (2017). Attention capture by sudden and unexpected changes: a
multisensory perspective (PhD dissertation). Umeå University, Umeå.
Retrieved from
\url{http://urn.kb.se/resolve?urn=urn:nbn:se:umu:diva-141852}

Marsja, E., Marsh, J. E., Hansson, P., \& Neely, G. (2019). Examining
the Role of Spatial Changes in Bimodal and Uni-Modal To-Be-Ignored
Stimuli and How They Affect Short-Term Memory Processes. Frontiers in
Psychology, 10. \url{https://doi.org/10.3389/fpsyg.2019.00299}

Marsja, E., Neely, G., and Ljungberg, J. K. (Under Review). Deviance
distraction in the auditory and tactile modalities after repeated
exposure: differential aspects of tactile and auditory deviants.

Rosa, E., Marsja, E., \& Ljungberg, J. K. (2020). Exploring Residual
Capacity: The Effectiveness of a Vibrotactile Warning During Increasing
Levels of Mental Workload in Simulated Flight Tasks. Aviation Psychology
and Applied Human Factors, 10(1), 13--23.
\url{https://doi.org//10.1027/2192-0923/a000180}

Stenbäck, V., Marsja, E., Hällgren, M., Lyxell, B., \& Larsby, B.
(Accepted). The contribution of age, working memory capacity and
inhibitory control on speech-recognition-in-noise in young, and older
adult listeners. Journal of Speech, Language, and Hearing Research.

Marsja, E., Stenbäck, V., Moradi, S., Danielsson, H., \& Rönnberg, J.
(Under Revision). Is Having Hearing Loss Fundamentally different?
Multi-group structural equation modeling of the effect of cognitive
functioning on speech identification.

Stenbäck, V., Marsja, E., Ellis, R., \& Rönnberg, J. (Under Review).
Relationships between objective and subjective outcome measures of
speech recognition in noise.

\end{document}