\documentclass[]{article}
\usepackage{pdflscape}
\usepackage{longtable}
\usepackage{booktabs}
\usepackage{everypage}

\newcommand{\Lpagenumber}{\ifdim\textwidth=\linewidth\else\bgroup
  \dimendef\margin=0 %use \margin instead of \dimen0
  \ifodd\value{page}\margin=\oddsidemargin
  \else\margin=\evensidemargin
  \fi
  \raisebox{\dimexpr -\topmargin-\headheight-\headsep-0.5\linewidth}[0pt][0pt]{%
    \rlap{\hspace{\dimexpr \margin+\textheight+\footskip}%
    \llap{\rotatebox{90}{ CV  - Erik Marsja 
- \thepage\hspace{1pt}(\pageref{LastPage})}}}}%
\egroup\fi}
\AddEverypageHook{\Lpagenumber}%

\usepackage{adjustbox}
\usepackage{titling}
\usepackage{caption}
\usepackage{tabularx}
\usepackage{tikz}
\usepackage{xcolor}
\usepackage{fancyhdr}
\usepackage{lastpage}
\usepackage{titlesec}
\usepackage[scaled=.90]{helvet}% Helvetica, served as a model for arial

\pagestyle{fancy}
\fancyhf{}
\renewcommand{\headrulewidth}{0pt}
\renewcommand{\footrulewidth}{0.0pt}
\fancyfoot[CO,CE]{   Redog\"{o}relse f\"{o}r vetenskaplig verksamhet \\  Erik Marsja -  \thepage\hspace{1pt}(\pageref{LastPage})}
\fancypagestyle{plain}{\pagestyle{fancy}}

\usepackage[margin=1.2in]{geometry}

% Font awesome
\usepackage{fontawesome}

\usepackage[T1]{fontenc}
\usepackage[utf8]{inputenc}

% For tables


% Tightlist

\providecommand{\tightlist}{%
  \setlength{\itemsep}{0pt}\setlength{\parskip}{0pt}}

% URLS
\usepackage[hidelinks]{hyperref}
\usepackage{breakurl}
\usepackage{float} % here for H placement parameter

% Appendix header style

\fancypagestyle{style2}{
\fancyhf{}
\fancyhead[C]{Appendix 1}
}


% For Changing the margins (Education)

\def\changemargin#1#2{\list{}{\rightmargin#2\leftmargin#1}\item[]}
\let\endchangemargin=\endlist 

%For displaying the table in landscape format
\usepackage[absolute]{textpos}

\fancypagestyle{lscape}{% 
\fancyhf{} % clear all header and footer fields 
\fancyfoot[LE]{%
\begin{textblock}{20}(1,5){\rotatebox{90}{\leftmark}}\end{textblock}
\begin{textblock}{1}(13,10.5){\rotatebox{90}{\thepage}}\end{textblock}}
\fancyfoot[LO] {%
\begin{textblock}{1}(13,10.5){\rotatebox{90}{\thepage}}\end{textblock}
\begin{textblock}{20}(1,13.25){\rotatebox{90}{\rightmark}}\end{textblock}}
\renewcommand{\headrulewidth}{0pt} 
\renewcommand{\footrulewidth}{0pt}}

\setlength{\TPHorizModule}{1cm}
\setlength{\TPVertModule}{1cm}

% Appendix
\fancypagestyle{style2}{
\fancyhf{}
\fancyhead[C]{Appendix 1}
\renewcommand{\headrulewidth}{1pt}
}



\newcommand\secbar {
    \tikz[baseline, trim left=3.2cm] 
    {
        \fill [white] (3cm,0.1ex) rectangle +(0.2cm,1.1ex);
        \draw [gray!95, fill=gray!80] (0cm,0.1ex) rectangle (3cm,1.1ex);        
    }
}
\newcommand\subsecbar {
    \tikz[baseline, trim left=0.15cm] 
    {
        \fill [white] (2cm,0.1ex) rectangle +(0.2cm,1.1ex);
        \fill [blue!40] (0cm,0.1ex) rectangle (2cm,1.1ex);      
    }
}

\newcommand\subsubsecbar {
    \tikz[baseline, trim left=0.15cm] 
    {
        \fill [white] (1cm,0.1ex) rectangle +(0.2cm,1.1ex);
        \fill [blue!40] (0cm,0.1ex) rectangle (2cm,1.1ex);      
    }
}

\titleformat{\section}{\large}{}{0cm}{\secbar}
\titleformat{\subsection}{\large}{}{0cm}{\normalfont\sffamily\Large\bfseries\subsecbar}
\titleformat{\subsubsection}{}{}{0cm}{\normalfont\sffamily\large\bfseries}

% No first line paragraph indent
\usepackage{parskip}
\usepackage{enumitem}


\titlespacing\section{0pt}{12pt plus 4pt minus 2pt}{4pt plus 2pt minus 2pt}
\titlespacing\subsection{0pt}{12pt plus 4pt minus 2pt}{4pt plus 2pt minus 2pt}
\titlespacing\subsubsection{0pt}{12pt plus 4pt minus 2pt}{4pt plus 2pt minus 2pt}

\newcolumntype{Y}{>{\centering\arraybackslash}X}

\begin{document}

\centerline{\huge \textbf{Erik Marsja} | \textcolor{darkgray}{Nuvarande och Planerad Forskning}}

\vspace{2 mm}

\hrule

\begin{table}[h]
\centering
\begin{tabularx}{\textwidth}{@{}lYl@{}}
\textbf{Home Address}: & &  \textbf{Personal Identity Number:} 
\\Tvistevägen 26, SE-907 36 Umeå, Sweden & &  19810526 
\\\\

 \faPhone \hspace{1 mm}  0046703633662  \hspace{1 mm}  &  & \faEnvelopeO \hspace{1 mm} \href{mailto:}{\tt \href{mailto:erik.marsja@liu.se}{\nolinkurl{erik.marsja@liu.se}}} \hspace{1 mm}  \\
 \faGlobe \hspace{1 mm} \href{http://www.marsja.se}{\tt www.marsja.se}   &  & \faGithub \hspace{1 mm} \href{http://github.com/marsja}{\tt marsja} \hspace{1 mm}  \\
 \multicolumn{3}{c}{\emph{Languages: }Swedish, English}
 \\\hline
\end{tabularx}
\end{table}

\hypertarget{tidigare-och-nuvarande-forskning}{%
\subsection{Tidigare och Nuvarande
Forskning}\label{tidigare-och-nuvarande-forskning}}

Min tidigare och nuvarande forskning har undersökt hur stimuli i
bakgrunden (exv. irrelevanta ljud eller vibrationer) påverkar mänsklig
perception. Från min kandidatuppsats, och fram till min
doktorsavhandling, studerade jag hur plötsliga förändringar, och
irrelevanta, i omgivningen fångar uppmärksamheten från en primär
uppgift. Under min tid som postdoktor har min forskning breddats till
att även innefatta funktionsnedsättning (än så länge hörselnedsättning),
kognitiv funktion, åldrande, och identifikationen av tal i bakgrundsbrus
och tal. Se följande två underrubriker för en mer detaljerad beskrivning
av min tidigare och nuvarande forskning.

\hypertarget{multisensorisk-perception-och-uppmuxe4rksamhet}{%
\subsubsection{Multisensorisk perception och
uppmärksamhet}\label{multisensorisk-perception-och-uppmuxe4rksamhet}}

I min kandidatuppdats undersökte jag huruvida det egna namnet har en
särskild förmåga att fånga uppmärksamhet i jämfört med ett annat,
familjärt, namn. Med detta experiment visade jag att så är inte fallet
när vi applicerar en gedigen experimentell kontroll (exv. matchar antal
stavelser i namnen). Vi följde sedan upp detta med ett experiment där vi
använda ord på vanliga objekt (exv. stol, och bord) och, återigen, så
visade vi att en persons eget namn har inte förmågan att fånga
uppmärksamheten mer än andra ord (Ljungberg et al., 2014). Dessa
resultat antyder att det finns begränsningar i de kognitiva teorier som
föreslagit att en persons egna namn exv. bryter igenom
uppmärksamhetsfiltret lättare än andra ord (e.g., Cherry, 1952).

I min doktorsavhandling (Marsja, 2017) undersökte jag hur irrelevanta
oväntade hörsel- och taktila stimuli har inverkan på visuell
bearbetning. I de två första studierna av min avhandling använde jag
enkla visuella kategoriseringsuppgifter och dessa två studier
identifierade en kunskapslucka som leder till den tredje studien: hur
påverkar plötsliga förändringar i irrelevanta auditiva, taktila eller
bimodala (både taktila och auditiva) sekvenser korttidsminnesprocesser?
Resultaten visade att distraktion av plötsliga, och oväntade, vibration
liknar distraktion som uppstår när plötsliga ljud presenteras
(cf.~Fabrice, 2014). Vidare fann jag att en möjlig skillnad är att
effekten av oväntade vibrationer försvinner över tid (Marsja, Neely,
K-Ljungberg, In Preparation). I den andra studien i avhandlingen fann
jag att ljud som presenteras bland upprepade vibrationer bara är
distraherande när vibrationen inte presenteras samtidigt som ljudet. Att
utelämna en upprepad vibration räcker för att fånga uppmärksamheten
(Marsja, Neely, K-Ljungberg, 2018). När det gäller bearbetning av
information i korttidsminnet visade resultaten att en plötslig
förändring av spatial lokalisation i irrelevant sekvens stör
korttidsminnesprocesser när den irrelevanta sekvensen består av både
ljud och vibrationer (Marsja, Marsh, Hansson, \& Neely, 2019).

Tillsammans med internationella och nationella forskare inom
kognitionspyskologi genomförde jag en serie om 3 experiment med syftet
att undersöka vad som stör det visuella korttidsminnet (Marsh, et al.,
In Preparation). I den här studien använde vi vibrotaktila sekvenser med
en konstant förändring (sekvensen ``hoppar'' mellan de två händerna,
vänster-höger-vänster-höger-vänster-höger; \emph{changing-state}
sekvens), och \emph{steady-state} sekvenser (när alla vibrationer i
sekvensen presenteras till båda händerna). Vi fann att visuellt
korttidsminne störs mer av en vibrotaktil sekvens som är changing-state
jämfört med en taktil sekvens som är steady-state, något som är likt
samma effekt för ljudsekvenser (Experiment 1); denna störning tycks
påverka återkallandet av objektens ordning i korttidsminnet (Experiment
2), och förutsägbarheten av vibrotaktila stimuli verkar inte påverka
omfattningen av effekten (Experiment 3). Studierna i min avhandling och
den tillsammans med Marsh et al.~(In Preparation) sammanfaller med
tidigare forskning (e.g., fann denna avhandling att både plötsliga och
oväntade förändringar i auditiva och taktila irrelevanta sekvenser
fångar uppmärksamhet från en visuell uppgift. Men den temporala
dynamiken mellan de två modaliteterna verkar skilja sig. Det verkar som
att upprepad utsättning för plötsliga och oväntade vibrationer gör att
den negativa effekten minskar, medan detta inte är fallet i den auditiva
modaliteten. Detta tyder på att det finns centrala mekanismer (upptäckt
av en plötsligförändring) och sensorikspecifika mekanismer.

Vidare har vi studerat interaktionen mellan de auditiva, visuella och
taktila modaliteterna från ett tillämpat perspektiv. I en studie syftade
vi till att fastställa hur effektiv en taktil varning är under ökande
nivåer av mental arbetsbelastning i en primär uppgift. I denna studie
använde vi tre simulerade flyguppgiftsförhållanden där vi varierade den
mentala arbetsbelastningen medan vi presenterade en på vibrotaktil
varning. Generellt sett så visade studien en minskad övergripande effekt
av varningssignalen när den mentala belastningen ökade. Dock så avtog
denna tendens allt eftersom nivån på arbetsbelastningen steg. Detta
indikerar att vibrotaktila varningssignaler kan användas för att
förmedla information under ökande nivåer av primär mental
arbetsbelastning (Rosa, Marsja, \& K-Ljungberg, 2020).

\hypertarget{tal-i-brus-och-huxf6rselnedsuxe4ttning}{%
\subsubsection{Tal i Brus och
hörselnedsättning}\label{tal-i-brus-och-huxf6rselnedsuxe4ttning}}

På senare tid har jag utökat min forskning till att innefatta området
kognitiv hörselvetenskap och handikappvetenskap. Tillsammans med
nationella forskare har vi fokuserat på sambandet mellan kognitiv
funktion, ålder och tal i brus. I en studie har vi använt avancerade
statistiska metoder (exv. faktoranalys och strukturell
ekvationsmodellering; SEM) med ett relativt stort stickprov för att
undersöka hur kognitiv funktion och ålder påverkar prestation i
tal-i-brus-test hos individer med och utan hörselnedsättning. Målet med
denna studie var att med hjälp av multigrupp-SEM modellera sambandet
mellan kognitiv funktion, åldrande, och identifikation av tal i brus.
Denna studie utökar tidigare studier som framförallt undersökt dessa
samband hos individer med hörselnedsättning, i mindre stickprov, och med
mindre avancerade statistiska metoder (se exv. Dryden et al., 2017 för
en översikt). Vi fann kognitiv funktion och ålder har liknande inverkan
oavsett om individen har en hörselnedsättning eller inte (Marsja et al.,
2022). Det vill säga, vi visade att för äldre individer är kognitiv
förmåga minst lika viktig för att identifiera tal i brus och detta är i
linje med Ease-of-Language Use-modellen (ELU; e.g., Rönnberg et al.,
2021). Enligt ELU antas det att både en process- och lagringsfunktion
krävs när det är ogynnsamma förhållanden, som exv. vid bakgrundsljud.
Detta resultat stämmer överens med vad vi funnit i en tidigare studie:
hög arbetsminneskapacitet är relaterad till bättre prestanda vid
taligenkänning i bakgrundsbrus för en grupp äldre och en grupp yngre
personer med normal hörsel (Stenbäck, et al., 2021). I denna studie fann
vi även att högre arbetsminneskapacitet var negativt relaterat till hur
ansträngande det var att lyssna på tal i bakgrundsbrus. Det vill säga,
högre arbetsminneskapacitet gör det lättare att lyssna (Stenbäck et al.,
2021).

Arbetsminneskapacitet och kognitiv kontroll (inhibition) har vi även
studerat i relation till bakgrundsljud som bär information eller inte
(Stenbäck et al., Submitted). Vi fann att arbetsminnet är viktigare när
bakgrundsbruset bär information (dvs. består av talljud) än när det inte
består av information (dvs. består statiskt brus). I en ytterligare
studie, hade vi som mål att undersöka samband mellan självrapporterade
hörselrelaterade mått (dvs. ett frågeformulär) och prestationsbaserade
beteendemått (taligenkänning i brus; Stenbäck et al., Under Revision).
Vi analyserade data från två tal-i-brustest som vanligen används i
hörselkliniken (Hagerman och Hearing-In-Noise-Test; HINT). Dessa två
tal-i-brustest skiljer sig i att meningarna i Hagerman saknar kontext
medan meningarna i HINT har kontext. Denna studie visade att beteende-
och självrapporteringsmått är relaterade till varandra hos äldre normalt
hörande vuxna. Vi fann emellertid inte ett samband mellan dessa
mätningar hos hörapparatanvändare, vilket belyser det tvetydiga
sambandet mellan beteenden. och självrapporteringsåtgärder. Slutligen
undersöker vi också hur logisk slutledning (både auditiv och visuell)
kan vara relaterad till tal i brusigenkänning (Stenbäck et al., In
Preparation).

\hypertarget{planerad-forskning}{%
\subsection{Planerad forskning}\label{planerad-forskning}}

Majoriteten av min nuvarande forskning har bedrivits, eller bedrivs,
genom att analysera data från ett longitudinellt projekt med, bland
annat, rad olika kognitions- och hörselrelaterade data (n200; Rönnberg
et al.~2016). En del av min planerade forskning är att följa upp
analyser för att även undersöka hur sambandet mellan ålder, hörsel, och
kognition ser ut över tid. Jag har, förstås, även utvecklat egna
projektidéer (se nedan). I de planerade projekten nedan är det planerat
att jag är forskningsledare om inte annat anges (dvs., om någon annan
anges som ``PI'' är det inte jag som är projektansvarig).

\hypertarget{zoom-fatigue}{%
\subsubsection{``Zoom Fatigue''}\label{zoom-fatigue}}

Tack vare min tidigare och nuvarande forskning är jag införstådd med att
hörselnedsättning påverkar förmågan att bland annat kommunicera och
etablera och upprätthålla sociala relationer. Tillsammans med forskare i
funktionsnedsättning, pedagogik, och audiologi så planerar jag att
undersöka hur hjälpmedel såsom textning av videosamtal kan påverka
förståelse och lyssningsansträngning. Generellt sett är lyssnande en
automatisk och enkel process, särskilt under ideala förhållanden där
inga explicita processer är engagerade (e.g., Rönnberg et al., 2013).
Men under ogynnsamma förhållanden (e.g., när talsignalen är förvrängd,
eller i situationer med mycket bakgrundsbrus) kan lyssnandet bli mer
ansträngande (Rönnberg et al., 2013). Ogynnsamma förhållanden uppstår i
både det sociala livet och arbetslivet. Till exempel, när man använder
digitala medier där talsignalen alltid är förvrängd (e.g., videosamtal
och möten). Dessutom är lyssnande betydligt mer ansträngande för
personer med hörselnedsättning (Kramer et al., 2006) och en stor del av
Sveriges befolkning i arbetsför ålder (18,5 \%) beräknas ha en
hörselnedsättning (Hörselskadades riksförbund, 2017).

Ogynnsamma lyssningsförhållanden kan också minska förståelsen och
påverka minnet negativt (e.g., för noveller; Piquado et al.~2012; Ward
et al.~2016). Till exempel rapporterade Piquado med kollegor (2012) att
lyssnare med hörselnedsättning visade sämre minne för noveller jämfört
med normalhörande lyssnare. På grund av ökad lyssningsansträngning kan
en individ med hörselnedsättning sluta använda hörapparater eller välja
att inte delta i ett socialt evenemang (se Peelle, 2018). Dessutom, även
vid deltagande, ökar ansträngande lyssnande stress och trötthet hos
individer med hörselnedsättning (Hornsby, 2013) och ökar frånvaro och
sjukdom på arbetsplatsen (e.g., Nachtegaal, 2009). Flera faktorer
angående hinder i arbetslivet har också identifierats för individer med
hörselnedsättning, inklusive brist på anpassning på arbetsplatsen och
brist på hjälpmedel (Granberg \& Gustafson, 2021). Det är därför av stor
vikt när stora förändringar görs inom arbetsmiljön att stor noggrannhet
iakttas för att möjliggöra anpassning och användning av hjälpmedel
(e.g., att använda textning i videomöten). Detta både för personer med
hörselnedsättning såväl som normalt hörande.

\hypertarget{studie-1-intervjuer-med-fokusgrupper-finansierad}{%
\paragraph{Studie 1: Intervjuer med fokusgrupper
(Finansierad)}\label{studie-1-intervjuer-med-fokusgrupper-finansierad}}

En kvalitativ delstudie av detta projekt är redan finansierad av
Hörselforskningsfonden (DNR: IBL-2021-00170) och har som mål att
beskriva upplevelser, förståelse, och konsekvenser av digitalisering av
arbetsliv. Vidare kommer resultatet från det finansierade projektet
också att vara en utgångspunkt för vidare forskning om förutsättningarna
för personer med hörselnedsättning i en digital värld samt en studie om
hur man kan förbättra situationen. Detta både genom att vi får en rik
bild, genom kvalitativ metod, och eftersom resultatet ämnas användas
till att skapa ett instrument som kan användas att undersöka detta i
även en kvantitativ kontext (se Studie 2 och 3, nedan).

\hypertarget{studie-2-och-3-fuxf6rbuxe4ttra-digitala-muxf6ten}{%
\paragraph{Studie 2 och 3: Förbättra digitala
möten}\label{studie-2-och-3-fuxf6rbuxe4ttra-digitala-muxf6ten}}

I detta projekt, planerar jag ytterligare två delstudier: en
experimentell studie där vi vill undersöka proportionen av fel i en text
(i relation till det talade ljudet) som krävs för att minska
lyssningsansträngning och öka förståelse av det som sägs. Eftersom
ljudkvalitet har funnits påverka både lyssningsansträngning och
förståelse (se Mattys et al., 2012 för en översikt) samt att videosamtal
medför föränderlig kvalitet så vill vi även undersöka detta i
experimentet. I den andra studien planerar vi att följa upp experimentet
med en enkätundersökning. Enkäten ska utformas baserat på de resultat vi
erhåller i det projekt som redan är finansierat. Denna ansats gör att vi
att vi även fångar upp mer hälso- och arbetspsykologiska aspekter (exv.
stress, deltagande i arbetslivet) av digitalisering samt hur textningens
eventuella effekter. Slutligen för att ytterligare fånga andra aspekter
såsom upplevelser och insikter gällande lyssningsansträngning och
förståelse, digitalisering, så ämnar vi att även följa upp med
intervjuer. Således har den andra planerade studien en
mixed-methods-ansats där vi integrerar kvalitativa data efter vi samlat
in kvantitativt (dvs. med enkäten). Vi kommer därmed erhålla en bred
bild av både positiva och negativa aspekter samt få ett resultat som vi
anser får en högre ekologisk validitet.

De två delstudierna (2 och 3), och studien finansierad av
Hörselforskningsfonden (Studie 1), bedrivs i samarbete med de
Hörselskadades Riksförbund (HRF). Jag planerar att söka medel för Studie
2 och 3 av projektet från bland annat FORTE och AFA försäkringar. Målet
med det övergripande projektet är att undersöka hur digitaliseringen av
arbete och det sociala live har påverkat individer med
hörselnedsättning. Specifikt syftar vi till att 1) undersöka om textstöd
i videomöten positivt påverkar förståelse och lyssnande, och 2) hur de
positiva effekterna av att använda textstöd kommer att påverka
delaktighet (både socialt och i arbetslivet).

Samarbetspartners:

\begin{itemize}
\tightlist
\item
  Dr Victoria Stenbäck, Avdelningen för Pedagogik och Didaktik,
  Institutionen för Beteendevetenskap och Lärande, Linköpings
  Universitet
\item
  Dr Carine Signoret, Avdelningen för Handikappvetenskap, Institutionen
  för Beteendevetenskap och Lärande, Linköpings Universitet
\item
  Dr Ann-Charlotte Bivall, Avdelningen för Pedagogik och Sociologi,
  Institutionen för Beteendevetenskap och Lärande, Linköpings
  Universitet
\item
  Dr Antje Heinrich, Avdelningen för Mänsklig Kommunikation, Utveckling
  och Hörsel, Manchester Universitet
\end{itemize}

\hypertarget{ytterligare-delstudier}{%
\paragraph{Ytterligare delstudier}\label{ytterligare-delstudier}}

Ovanstående projekt ämnas att utföras med en än mer experimentell än
tillämpad ansats där vi endast fokuserar på experimentella
manipulationer av textningen. Detta projekt ämnar bidra till teoretisk
kunskap kring audiovisuellt processande hos individer med
hörselnedsättning såväl som med andra funktionsnedsättningar men även
normalt hörande individer. Detta projekt är än i sin linda men vi har
identifierat lämpliga finansiärer som till exempel Vetenskapsrådet,
Riksbankens Jubileumsfond.

Samarbetspartners:

\begin{itemize}
\tightlist
\item
  Dr Victoria Stenbäck, Avdelningen för Pedagogik och Didaktik,
  Institutionen för Beteendevetenskap och Lärande, Linköpings
  Universitet
\item
  Dr Carine Signoret, Avdelningen för Handikappvetenskap, Institutionen
  för Beteendevetenskap och Lärande, Linköpings Universitet
\end{itemize}

\hypertarget{relationen-mellan-luxe4s--och-skrivvanor-och-psykosocial-huxe4lsa-hos-skolungdomar-med-huxf6rselnedsuxe4ttning}{%
\subsubsection{Relationen mellan läs- och skrivvanor och psykosocial
hälsa hos skolungdomar med
hörselnedsättning}\label{relationen-mellan-luxe4s--och-skrivvanor-och-psykosocial-huxe4lsa-hos-skolungdomar-med-huxf6rselnedsuxe4ttning}}

I detta projekt planerar vi att undersöka ungdomar med hörselnedsättning
i årskurs 7--9 och om läs- och skrivvanor, motivation, betyg och
stödundervisning samt psykosocial hälsa. Framgångsrik
talspråksutveckling hos barn med hörselnedsättning är beroende av tidig
upptäckt av hörselnedsättningen, tillgång till hörselhjälpmedel -- till
exempel hörapparat eller cochleaimplantat -- och kontinuerlig
intervention med syfte att främja användningen av talat språk (e.g.,
Yoshinaga-Itano et al., 2017). Trots medicinska, teknologiska och
pedagogiska framsteg, medför emellertid en hörselnedsättning, oavsett
svårighetsgrad eller typ, likväl en ökad risk för problem med det talade
språket (Tomblin et al., 2015), och svårigheter med läsning och
skrivning som kräver uppföljning och intervention under skolåren (Wang
et al., 2019). Elever med hörselnedsättning ligger ofta efter jämnåriga
med typisk hörsel gällande förmågor relaterade till språk, läsning och
skrivning, till exempel talspråksförståelse, läsförståelse och förmågan
att uttrycka sig i skrift (Sarant et al 2015). Elever med
hörselnedsättning når inte läroplanens kunskapsmål i samma utsträckning
som jämnåriga, och andelen som går vidare till högskolestudier är bara
omkring 15 procent (SOU, 2016), jämfört med över 50 procent för elever
med typisk hörsel (SCB, 2021). Hörselnedsättning ökar också risken att
drabbas av psykosocial ohälsa (Theunissen et al., 2014).

Målet med detta projekt är att öka kunskapen om hur, vad och varför
ungdomar med hörselnedsättning läser och skriver. Det gäller till
exempel hur mycket tid som tillbringas med att läsa och skriva och i
vilka sammanhang det görs (hur), vilken typ av texter som läses (vad)
och vad som motiverar till läsning och skrivning (varför). Ett annat mål
är att ge en bild av vilket stöd ungdomar med hörselnedsättning får
eller upplever sig behöva i skolan för att kunna delta på lika villkor i
aktiviteter som inkluderar läsning och skrivning. Ytterligare ett mål är
att få förståelse för hur läs- och skrivvanor hänger samman med hur
eleverna mår och hur väl de uppnår kunskapsmålen i skolan. Projektet
kommer att ge kunskap om vad som hindrar respektive underlättar
framgångsrik och lustfylld läsning och skrivning. Med den kunskapen som
grund kan sedan konkreta förslag ges på hur läsning och skrivning kan
främjas i denna grupp.

Samarbetsparnters:

\begin{itemize}
\tightlist
\item
  Dr Victoria Stenbäck (PI), Avdelningen för Pedagogik och Didaktik,
  Institutionen för Beteendevetenskap och Lärande, Linköpings
  Universitet
\item
  Dr Simon Sundström, Institutet för Specialpedagogik, Oslo Universitet
\end{itemize}

\hypertarget{referenser}{%
\subsubsection{Referenser}\label{referenser}}

Cherry, E. C. (1953). Some experiments on The recognition of Speech,
with one and with Two Ears. \emph{The Journal of the Acoustical Society
of America, 25}(5), 975--979.

Dryden, A., Allen, H. A., Henshaw, H., \& Heinrich, A. (2017). The
Association Between Cognitive Performance and Speech-in-Noise Perception
for Adult Listeners: A Systematic Literature Review and Meta-Analysis.
Trends in Hearing, 21, 1--21.
\url{https://doi.org/10.1177/2331216517744675}

Granberg, S., \& Gustafsson, J. (2021). Key findings about hearing loss
in the working-life: a scoping review from a well-being perspective.
\emph{International Journal of Audiology, 60}(sup2), 60--70.
\url{https://doi.org/10.1080/14992027.2021.1881628}

Hornsby, B. W. Y. (2013). The effects of hearing aid use on listening
effort and mental fatigue associated with sustained speech processing
demands. \emph{Ear and Hearing, 34}(5), 523--534.
\url{https://doi.org/10.1097/AUD.0b013e31828003d8}

Hörselskadades Riksförbund. (2017). Hörselskadade i siffror 2017.
Stockholm.

Kramer, S. E., Kapteyn, T. S., \& Houtgast, T. (2006). Occupational
performance: Comparing normally-hearing and hearing-impaired employees
using the Amsterdam Checklist for Hearing and Work. \emph{International
Journal of Audiology, 45}(9), 503--512.
\url{https://doi.org/10.1080/14992020600754583}

Ljungberg, J. K., Parmentier, F. B. R., Jones, D. M., Marsja, E., \&
Neely, G. (2014). `What's in a name?' `No more than when it's mine own'.
Evidence from auditory oddball distraction. \emph{Acta Psychologica,
150}, 161--166. \url{https://doi.org/10.1016/j.actpsy.2014.05.009}

Ljungberg, J. K., \& Parmentier, F. B. R. (2012). Cross-modal
distraction by deviance: Functional similarities between the auditory
and tactile modalities. Experimental Psychology, 59(6), 355--363.
\url{https://doi.org/10.1027/1618-3169/a000164}

Marsh, J. E., Vachon, F., Sörqvist, P., Marsja, E., Röer, J. P., and
Ljungberg, J. K. (Manuscript in Preparation). Irrelevant vibro-tactile
stimuli produce a changing-state effect: Implications for theories of
interference in short-term memory.

Marsja, E., Neely, G., and Ljungberg, J. K. (2018). Investigating
Deviance Distraction and the Impact of the Modality of the To-Be-Ignored
Stimuli. \emph{Experimental Psychology 65}.
\url{doi:10.1027/1618-3169/a000390}.

Marsja, E. (2017). Attention capture by sudden and unexpected changes: a
multisensory perspective (PhD thesis). Umeå University, Umeå. Retrieved
from \url{http://urn.kb.se/resolve?urn=urn:nbn:se:umu:diva-141852}

Marsja, E., Marsh, J. E., Hansson, P., \& Neely, G. (2019). Examining
the Role of Spatial Changes in Bimodal and Uni-Modal To-Be-Ignored
Stimuli and How They Affect Short-Term Memory Processes. \emph{Frontiers
in Psychology}, 10. \url{https://doi.org/10.3389/fpsyg.2019.00299}

Marsja, E., Neely, G., and Ljungberg, J. K. (In Preparation). Deviance
distraction in the auditory and tactile modalities after repeated
exposure: differential aspects of tactile and auditory deviants.

Marsja, E., Stenbäck, V., Moradi, S., Danielsson, H., \& Rönnberg, J.
(Accepted). Is Having Hearing Loss Fundamentally different? Multi-group
structural equation modeling of the effect of cognitive functioning on
speech identification. \emph{Ear and Hearing}.

Mattys, S. L., Davis, M. H., Bradlow, A. R., \& Scott, S. K. (2012).
Speech recognition in adverse conditions: A review. \emph{Language and
Cognitive Processes, 27}(7--8), 953--978.
\url{https://doi.org/10.1080/01690965.2012.705006}

Nachtegaal, J., Kuik, D. J., Anema, J. R., Goverts, S. T., Festen, J.
M., \& Kramer, S. E. (2009). Hearing status, need for recovery after
work, and psychosocial work characteristics: Results from an
internet-based national survey on hearing. \emph{International Journal
of Audiology, 48}(10), 684--691.
\url{https://doi.org/10.1080/14992020902962421}

Peelle, J. E. (2018). Listening effort: How the cognitive consequences
of acoustic challenge are reflected in brain and behavior. \emph{Ear and
Hearing, 39}(2), 204--214.
\url{https://doi.org/10.1097/AUD.0000000000000494}

Piquado, T., Benichov, J. I., Brownell, H., \& Wingfield, A. (2012). The
hidden effect of hearing acuity on speech recall, and compensatory
effects of self-paced listening. \emph{International Journal of
Audiology, 51}(8), 576--583.
\url{https://doi.org/10.3109/14992027.2012.684403}

Rosa, E., Marsja, E., \& Ljungberg, J. K. (2020). Exploring Residual
Capacity: The Effectiveness of a Vibrotactile Warning During Increasing
Levels of Mental Workload in Simulated Flight Tasks. \emph{Aviation
Psychology and Applied Human Factors, 10}(1), 13--23.
\url{https://doi.org//10.1027/2192-0923/a000180}

Rönnberg, J., Lunner, T., Zekveld, A., Sörqvist, P., Danielsson, H.,
Lyxell, B., Dahlström, Ö., Signoret, C., Stenfelt, S., Pichora-Fuller,
M. K., \& Rudner, M. (2013). The Ease of Language Understanding (ELU)
model: theoretical, empirical, and clinical advances. \emph{Frontiers in
Systems Neuroscience, 7}(JUNE), 1--17.
\url{https://doi.org/10.3389/fnsys.2013.00031}

Rönnberg, J., Lunner, T., Ng, E. H. N., Lidestam, B., Zekveld, A. A.,
Sörqvist, P., Lyxell, B., Träff, U., Yumba, W., Classon, E., Hällgren,
M., Larsby, B., Signoret, C., Pichora-Fuller, M. K., Rudner, M.,
Danielsson, H., \& Stenfelt, S. (2016). Hearing impairment, cognition
and speech understanding: exploratory factor analyses of a comprehensive
test battery for a group of hearing aid users, the n200 study.
International Journal of Audiology, 55(11), 623--642.
\url{https://doi.org/10.1080/14992027.2016.1219775}

Sarant, J. Z., Harris, D. C., \& Bennet, L. A. (2015). Academic outcomes
for school-aged children with severe--profound hearing loss and early
unilateral and bilateral cochlear implants. \emph{Journal of Speech,
Language, and Hearing Research, 58}(3), 1017--1032.

SCB. 2021. Övergång gymnasieskola -- eftergymnasial utbildning,
examinerade från gymnasieskolan

SOU 2016:46. 2016. Samordning, ansvar och kommunikation -- vägen till
ökad kvalitet i utbildningen för elever med vissa
funktionsnedsättningar. Utbildningsdepartementet.

Stenbäck, V., Marsja, E., Hällgren, M., Lyxell, B., \& Larsby, B.
(2021). The Contribution of Age, Working Memory Capacity, and Inhibitory
Control on Speech Recognition in Noise in Young and Older Adult
Listeners. \emph{Journal of Speech, Language, and Hearing Research,
64}(11), 4513--4523. \url{https://doi.org/10.1044/2021_JSLHR-20-00251}

Stenbäck, V., Marsja, E., Hällgren, M., Lyxell, B., \& Larsby, B.
(Submitted). Informational masking and listening effort in
speech-recognition-in-noise -- the role of working memory capacity and
inhibitory control in older adults with and without hearing impairment.

Stenbäck, V., Marsja, E., Ellis, R., \& Rönnberg, J. (Submitted).
Relationships between objective and subjective outcome measures of
speech recognition in noise.

Theunissen, S. C. P. M., Rieffe, C., Kouwenberg, M., De Raeve, L. J. I.,
Soede, W., Briaire, J. J., \& Frijns, J. H. M. (2014). \emph{Behavioral
problems in school-aged hearing-impaired children: the influence of
sociodemographic, linguistic, and medical factors. European Child \&
Adolescent Psychiatry, 23}(4), 187--196.

Tomblin, J. B., Harrison, M., Ambrose, S. E., Walker, E. A., Oleson, J.
J., \& Moeller, M. P. (2015). Language Outcomes in Young Children with
Mild to Severe Hearing Loss. \emph{Ear \& Hearing, 36}(Supplement 1),
76S-91S. \url{https://doi.org/10.1097/AUD.0000000000000219}

Wang, J., Quach, J., Sung, V., Carew, P., Edwards, B., Grobler, A.,
Gold, L., \& Wake, M. (2019). Academic, behavioural and quality of life
outcomes of slight to mild hearing loss in late childhood: a
population-based study. \emph{Archives of Disease in Childhood,
104}(11), 1056--1063.

Ward, C. M., Rogers, C. S., Van Engen, K. J., \& Peelle, J. E. (2016).
Effects of Age, Acoustic Challenge, and Verbal Working Memory on Recall
of Narrative Speech. \emph{Experimental Aging Research, 42}(1), 97--111.
\url{https://doi.org/10.1080/0361073X.2016.1108785}

Yoshinaga-Itano, C., Sedey, A. L., Wiggin, M., \& Chung, W. (2017).
Early Hearing Detection and Vocabulary of Children With Hearing Loss.
\emph{Pediatrics, 14}0(2). \url{https://doi.org/10.1542/peds.2016-2964}

\end{document}