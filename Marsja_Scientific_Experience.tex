\documentclass[]{article}
\usepackage{pdflscape}
\usepackage{longtable}
\usepackage{booktabs}
\usepackage{everypage}

\newcommand{\Lpagenumber}{\ifdim\textwidth=\linewidth\else\bgroup
  \dimendef\margin=0 %use \margin instead of \dimen0
  \ifodd\value{page}\margin=\oddsidemargin
  \else\margin=\evensidemargin
  \fi
  \raisebox{\dimexpr -\topmargin-\headheight-\headsep-0.5\linewidth}[0pt][0pt]{%
    \rlap{\hspace{\dimexpr \margin+\textheight+\footskip}%
    \llap{\rotatebox{90}{ CV  - Erik Marsja 
- \thepage\hspace{1pt}(\pageref{LastPage})}}}}%
\egroup\fi}
\AddEverypageHook{\Lpagenumber}%

\usepackage{adjustbox}
\usepackage{titling}
\usepackage{caption}
\usepackage{tabularx}
\usepackage{tikz}
\usepackage{xcolor}
\usepackage{fancyhdr}
\usepackage{lastpage}
\usepackage{titlesec}
\usepackage[scaled=.80]{helvet}% Helvetica, served as a model for arial

\pagestyle{fancy}
\fancyhf{}
\renewcommand{\headrulewidth}{0pt}
\renewcommand{\footrulewidth}{0.0pt}
\fancyfoot[CO,CE]{   Erik Marsja -  \thepage\hspace{1pt}(\pageref{LastPage})}
\fancypagestyle{plain}{\pagestyle{fancy}}

\usepackage[margin=1in]{geometry}

% Font awesome
\usepackage{fontawesome}

\usepackage[T1]{fontenc}
\usepackage[utf8]{inputenc}

% For tables


% Tightlist

\providecommand{\tightlist}{%
  \setlength{\itemsep}{0pt}\setlength{\parskip}{0pt}}

% URLS
\usepackage[hidelinks]{hyperref}
\usepackage{breakurl}
\usepackage{float} % here for H placement parameter

% Appendix header style

\fancypagestyle{style2}{
\fancyhf{}
\fancyhead[C]{Appendix 1}
}


% For Changing the margins (Education)

\def\changemargin#1#2{\list{}{\rightmargin#2\leftmargin#1}\item[]}
\let\endchangemargin=\endlist 

%For displaying the table in landscape format
\usepackage[absolute]{textpos}

\fancypagestyle{lscape}{% 
\fancyhf{} % clear all header and footer fields 
\fancyfoot[LE]{%
\begin{textblock}{20}(1,5){\rotatebox{90}{\leftmark}}\end{textblock}
\begin{textblock}{1}(13,10.5){\rotatebox{90}{\thepage}}\end{textblock}}
\fancyfoot[LO] {%
\begin{textblock}{1}(13,10.5){\rotatebox{90}{\thepage}}\end{textblock}
\begin{textblock}{20}(1,13.25){\rotatebox{90}{\rightmark}}\end{textblock}}
\renewcommand{\headrulewidth}{0pt} 
\renewcommand{\footrulewidth}{0pt}}

\setlength{\TPHorizModule}{1cm}
\setlength{\TPVertModule}{1cm}

% Appendix
\fancypagestyle{style2}{
\fancyhf{}
\fancyhead[C]{Appendix 1}
\renewcommand{\headrulewidth}{1pt}
}



\newcommand\secbar {
    \tikz[baseline, trim left=3.2cm] 
    {
        \fill [white] (3cm,0.1ex) rectangle +(0.2cm,1.1ex);
        \draw [gray!95, fill=gray!80] (0cm,0.1ex) rectangle (3cm,1.1ex);        
    }
}
\newcommand\subsecbar {
    \tikz[baseline, trim left=0.15cm] 
    {
        \fill [white] (2cm,0.1ex) rectangle +(0.2cm,1.1ex);
        \fill [blue!40] (0cm,0.1ex) rectangle (2cm,1.1ex);      
    }
}

\newcommand\subsubsecbar {
    \tikz[baseline, trim left=0.15cm] 
    {
        \fill [white] (1cm,0.1ex) rectangle +(0.2cm,1.1ex);
        \fill [blue!40] (0cm,0.1ex) rectangle (2cm,1.1ex);      
    }
}

\titleformat{\section}{\large}{}{0cm}{\secbar}
\titleformat{\subsection}{\large}{}{0cm}{\normalfont\sffamily\Large\bfseries\subsecbar}
\titleformat{\subsubsection}{}{}{0cm}{\normalfont\sffamily\large\bfseries}

% No first line paragraph indent
\usepackage{parskip}
\usepackage{enumitem}

\setlength{\parskip}{0cm}


\titlespacing\section{0pt}{12pt plus 4pt minus 2pt}{1pt plus 1pt minus 2pt}
\titlespacing\subsection{0pt}{12pt plus 4pt minus 2pt}{1pt plus 1pt minus 2pt}
\titlespacing\subsubsection{0pt}{12pt plus 4pt minus 2pt}{4pt plus 2pt minus 2pt}

\newcolumntype{Y}{>{\centering\arraybackslash}X}

\begin{document}

\centerline{\huge \textbf{Erik Marsja} | \textcolor{darkgray}{Nuvarande
och Planerad Forskning}}

\vspace{1mm}

\hrule

\begin{table}[h]
\centering
\begin{tabularx}{\textwidth}{@{}lYl@{}}
\textbf{Home Address}: & &  \textbf{Personal Identity Number:} 
\\Tvistevägen 26, SE-907 36 Umeå, Sweden & &  19810526 
\\\\

 \faPhone \hspace{1 mm}  +46703633662  \hspace{1 mm}  &  & \faEnvelopeO \hspace{1 mm} \href{mailto:}{\tt \href{mailto:erik.marsja@liu.se}{\nolinkurl{erik.marsja@liu.se}}} \hspace{1 mm}  \\
 \faGlobe \hspace{1 mm} \href{http://www.marsja.se}{\tt www.marsja.se}   &  & \faGithub \hspace{1 mm} \href{http://github.com/marsja}{\tt marsja} \hspace{1 mm}  \\
 \multicolumn{3}{c}{}
 \\\hline
\end{tabularx}
\end{table}

\hypertarget{tidigare-och-nuvarande-forskning}{%
\subsection{Tidigare och Nuvarande
Forskning}\label{tidigare-och-nuvarande-forskning}}

Min forskning har främst fokuserat på hur irrelevant stimuli påverkar
olika typer av prestation. Från min kandidatuppsats, och fram till min
avhandling, studerade jag hur plötsliga förändringar i omgivningen
fångar uppmärksamheten från en primär uppgift. Senare har jag breddat
mig till att forska på bl.a. funktionsnedsättning (främst
hörselnedsättning men snart även ADHD), kognitiv funktion, och åldrande.

\hypertarget{multisensorisk-perception-och-uppmuxe4rksamhet}{%
\subsubsection{Multisensorisk perception och
uppmärksamhet}\label{multisensorisk-perception-och-uppmuxe4rksamhet}}

I min kandidatuppsats fann vi att det egna namnet inte är mer
distraherande jämfört med ett familjärt namn. Detta följdes upp med ett
experiment där vi använde ord på vanliga objekt (exv. stol) och
resultaten från det första experimentet replikerades (Ljungberg et al.,
2014). I min avhandling (Marsja, 2017) undersökte jag hur irrelevanta
och oväntade auditiva och taktila stimuli fångar uppmärksamheten. Jag
fann att distraktion i den taktila modaliteten liknar den distraktion
den i den auditiva modaliteten men att en möjlig skillnad är att
effekten av oväntade vibrationer försvinner över tid (Marsja et al., In
Prep.). Vidare visade jag att ljud som presenteras bland upprepade
vibrationer bara är distraherande när vibrationen inte presenteras
samtidigt som ett ljud. Att utelämna en upprepad vibration räcker för
att påverka prestation (Marsja et al., 2018). En plötslig förändring av
spatial lokalisation i en, för den primära uppgiften, irrelevant sekvens
stör korttidsminnet men enbart när sekvensen består av både ljud och
vibrationer (Marsja, et al., 2019).

Tillsammans med nationella och internationella forskare undersökte vi
även likheter och skillnader i distraktion av ljud och vibrationer och
dess påverkan på korttidsminnet. Vi fann att korttidsminne störs mer av
en vibrotaktil sekvens som är föränderlig jämfört med en sekvens som är
samma hela tiden, ett resultat det som finns rapporterat för
ljudsekvenser (Marsh et al.~In Prep.). Tillsammans med forskare vid
Luleå Tekniska Universitet planerar vi en studie där vi ämnar att
jämföra vibrotaktila och auditiva oväntade uppgiftsirrelevanta
förändringar hos vuxna med ADHD och vuxna utan ADHD. Denna studie är
pågående och nyligen förregistrerad på Open Science Framework
(\url{https://osf.io/2uef8}). I en annan studie visade vi en minskad
effekt av varningssignalen när den mentala belastningen ökade men som
nådde en platå allt eftersom belastningen blev tillräckligt hög (Rosa,
\textbf{Marsja}, \& K-Ljungberg, 2020).

\hypertarget{huxf6rselnedsuxe4ttning---kognition-och-tal-i-brus}{%
\subsubsection{Hörselnedsättning - kognition och tal i
brus}\label{huxf6rselnedsuxe4ttning---kognition-och-tal-i-brus}}

Jag har undersökt hur kognitiv funktion och ålder är relaterat till
taligenkänning hos individer med och utan hörselnedsättning. Vi fann att
kognition och ålder har liknande inverkan oavsett hörselnedsättning
eller inte (\textbf{Marsja} et al., 2022). Detta korrobrerar med
resultat från i en annan studie: hög arbetsminneskapacitet är relaterad
till bättre prestanda vid taligenkänning i bakgrundsbrus för en grupp
äldre och en grupp yngre normalt hörande individer (Stenbäck, et al.,
2021). Här fann vi även att högre arbetsminneskapacitet var negativt
relaterat till hur ansträngande det var att lyssna på tal i
bakgrundsbrus.

Arbetsminne har vi funnit är viktigare när bakgrundsbruset bär
information (dvs. består av talljud) än när det inte består av
information (dvs. statiskt brus; Stenbäck et al., Submitted). I en av
våra studier undersökte vi samband mellan självrapporterade
hörselrelaterade mått, inklusive sociala aspekter, och beteendemått
(Stenbäck et al., 2022). Vi fann endast ett signifikant samband mellan
självskattningar och taligenkänning i brus hos äldre personer med normal
hörsel. Vidare undersöker vi också hur logisk slutledning (både auditiv
och visuell) kan vara relaterad till tal-i-brus-igenkänning (Stenbäck et
al., In Prep.). Andra studier som är planerade är att undersöka
dimensionerna i ett självskattninginstrument (SSQ; se Stenbäck et al.,
2022) med hjälp av exploratory graph analysis, en relativt ny och
avancerad statistisk metod. Vidare planerar vi att applicera
multigrupp-SEM för att undersöka sambandet mellan 1) kognitiv funktion
och olika nivåer av taltrösklar (eg.., 50\%, 80\%), och 2) huruvida
kognitiv funktion har en liknande relation till taligenkänning när
bakgrundsbruset är informativt eller när det inte är det, och hur detta
ser ut beroende på om det finns en

\hypertarget{planerad-forskning}{%
\subsection{Planerad forskning}\label{planerad-forskning}}

I de planerade projekten nedan är jag forskningsledare om inte annat
anges.

\hypertarget{att-fuxf6rbuxe4ttra-digitala-muxf6ten-fuxf6r-personer-med-huxf6rselnedsuxe4ttning}{%
\subsubsection{Att förbättra digitala möten för personer med
hörselnedsättning}\label{att-fuxf6rbuxe4ttra-digitala-muxf6ten-fuxf6r-personer-med-huxf6rselnedsuxe4ttning}}

Tillsammans med forskare i funktionsnedsättning och samhälle, pedagogik,
och audiologi så planerar jag att undersöka hur digitala möten upplevs
av samt hur, hjälpmedel såsom textning av videosamtal kan underlätta
arbetslivet för, anställda med hörselnedsättning.

\hypertarget{studie-1-intervjustudie-finansierad}{%
\paragraph{Studie 1: Intervjustudie
(Finansierad)}\label{studie-1-intervjustudie-finansierad}}

En delstudie av projektet är finansierad av Hörselforskningsfonden
(IBL-2021-00170) och har som mål att beskriva upplevelser, förståelse,
och konsekvenser av digitala möten. Resultatet från delstudien kommer
också att vara en utgångspunkt för vidare forskning om
förutsättningarna, och förbättringsmöjligheterna, för anställda med
hörselnedsättning.

\hypertarget{studie-2-och-3-fuxf6rbuxe4ttra-digitala-muxf6ten}{%
\paragraph{Studie 2 och 3: Förbättra digitala
möten}\label{studie-2-och-3-fuxf6rbuxe4ttra-digitala-muxf6ten}}

Detta projekt innefattar två delstudier: en experimentell studie med mål
att studera hur undertext kan påverka lyssningsansträngning och
förståelse av det som sägs under ett videomöte. I delstudie 2 vill vi
följa upp experimentet med en enkätundersökning som utformas baserat på
de resultat vi får från Studie 1 för att fånga mer hälso- och
arbetspsykologiska aspekter (exv. stress, deltagande i arbetslivet) av
digitalisering samt hur textningens eventuella effekter. Slutligen för
att fånga andra aspekter såsom upplevelser och insikter gällande
lyssningsansträngning och förståelse, digitalisering, och textning,
planerar att utföra intervjuer. Således har det större projektet en
mixed-methods-ansats där vi integrerar kvalitativa data efter vi samlat
in kvantitativt (dvs. med enkäten). Här planerar vi att använda
avancerade statistiska metoder som t.ex. exploratory graph analysis och
experimentet kommer förregistreras.

\hypertarget{projektgrupp}{%
\paragraph{Projektgrupp:}\label{projektgrupp}}

Dr Stenbäck, Avdelningen för Pedagogik och Didaktik (PEDI),
Institutionen för Beteendevetenskap och Lärande (IBL), Linköpings
Universitet, Dr Signoret, Avdelningen för Funktionsnedsättning och
Samhälle (FuSa), IBL, Linköpings Universitet, Dr Bivall, Avdelningen för
Pedagogik och Sociologi (APS), IBL, Linköpings Universitet, Dr Heinrich,
Avdelningen för Mänsklig Kommunikation, Utveckling och Hörsel,
Manchester Universitet. Projektet har skickats in till FORTE - Årliga
utlysningen (2022) och kommer att skickas in till AFA Försäkringar.

Fler experiment med fokus på manipulationer av textningen (exv. grad av
mismatch mellan tal och text) är även planerade. Dessa kan bidra till
teoretisk kunskap kring audiovisuellt processande hos individer med
hörselnedsättning. Här är främsta fokus på hörselnedsättning vill vi
senare bredda forskningen till att inkludera andra
funktionsnedsättningar (exv. kognitiva) och normalt hörande individer.
Potentiella finansiärer är Vetenskapsrådet (VR), FORTE, och AFA
Försäkringar.

\hypertarget{relationen-mellan-luxe4s--och-skrivvanor-och-psykosocial-huxe4lsa-hos-skolungdomar-med-huxf6rselnedsuxe4ttning}{%
\subsubsection{Relationen mellan läs- och skrivvanor och psykosocial
hälsa hos skolungdomar med
hörselnedsättning}\label{relationen-mellan-luxe4s--och-skrivvanor-och-psykosocial-huxe4lsa-hos-skolungdomar-med-huxf6rselnedsuxe4ttning}}

I detta projekt vill vi undersöka ungdomar med hörselnedsättning i
årskurs 4--9 gällande läs- och skrivvanor, betyg och stödundervisning
samt psykosocial hälsa. Syftet är att undersöka psykosocialt
välbefinnande i skolan hos barn och ungdomar med hörselnedsättning i
relation till pedagogiskt stöd och skolprestationer.

\hypertarget{projektgrupp-1}{%
\paragraph{Projektgrupp:}\label{projektgrupp-1}}

Dr Stenbäck (\emph{PI}), PEDI, IBL, Linköpings Universitet och Dr
Sundström, Institutet för Specialpedagogik, Oslo Universitet. Två
ansökningar med aningen olika fokus planeras att skickas in till VR
Utbildningsvetenskap (fokus på skolan och utbildningsmiljön) och FORTE -
Barns och Ungas psykiska hälsa (fokus på psykosocial hälsa).

\hypertarget{digital-tvilling---en-app-fuxf6r-att-underluxe4tta-resande-fuxf6r-personer-med-suxe4rskilda-behov}{%
\subsubsection{Digital Tvilling - En App för att Underlätta Resande för
Personer med Särskilda
Behov}\label{digital-tvilling---en-app-fuxf6r-att-underluxe4tta-resande-fuxf6r-personer-med-suxe4rskilda-behov}}

I detta projekt är vår målsättning att undersöka, och dokumentera,
barriärer samt behov som äldre och personer med funktionsnedsättning har
för att resa i kollektivtrafik. Resultatet planerar vi att använda för
att ta fram bland annat en kravlista för ett digitalt hjälpmedel som kan
användas av studiens målgrupp men även andra.

\hypertarget{projektgrupp-2}{%
\paragraph{Projektgrupp:}\label{projektgrupp-2}}

Dr Solis (\emph{PI}), Människan i Trafiksystemet, Statens Väg- och
Transportforskningsinstitut (VTI), Linköping, Sverige och Dr Nyberg,
Mobilitet, aktörer och planering, VTI. En skiss kommer skickas till
\href{https://www.k2centrum.se/}{K2 Kompetenscentrum} för forskning och
utbildning om kollektivtrafik (den 14:e april).

\hypertarget{den-tidsmuxe4ssiga-strukturen-bakom-nyfuxf6ddas-imitation}{%
\subsubsection{Den tidsmässiga strukturen bakom nyföddas
imitation}\label{den-tidsmuxe4ssiga-strukturen-bakom-nyfuxf6ddas-imitation}}

Här vill vi genom avancerade beräkningsmetoder (deep
learning/maskininlärning) kartlägga den temporala strukturen hos
neonatal imitation i videoinspelningar. Vi kommer att undersöka om
neonatal imitation kan förstås som en del av det svarsmönster som
spädbarn använder när de interagerar med en person. Vi förväntar oss att
framgångsrik imitation associeras med detta kommunikativa mönster medan
den misslyckade imitationsförsök gör det inte. Denna studie kommer att
ge unika data som gör det möjligt att förklara hur responser som används
som evidens för neonatal imitation är tidsmässigt organiserade. Det
kommer också att bidra till en bättre förståelse för spädbarns förmåga
att delta i målmedvetna utbyten med andra.

\hypertarget{projektgrupp-3}{%
\paragraph{Projektgrupp:}\label{projektgrupp-3}}

Prof.~Heimann (\emph{PI}), Avdelningen för Psykologi, IBL, Linköpings
Universitet, Dr Holmer, FuSa, IBL, Linköpings Universitet,
Prof.~Delafield-Butt, Akademin för Utbildning, Strathclyde Universitet,
Glasgow, Skottland, och Dr Tachtatzis, Institutionen för Elektronik och
elektroteknik, Strathclyde Universitet, Glasgow. Projektet är inskickat
som skiss till Riksbankens Jubileumsfond.

\hypertarget{referenser}{%
\subsubsection{Referenser}\label{referenser}}

\small

Ljungberg, J. K., Parmentier, F. B. R., Jones, D. M., \textbf{Marsja},
E., \& Neely, G. (2014). `What's in a name?' `No more than when it's
mine own'. Evidence from auditory oddball distraction. \emph{Acta
Psychol, 150}, 161--166.

Marsh, J. E., Vachon, F., Sörqvist, P., \textbf{Marsja}, E., Röer, J.
P., \& Ljungberg, J. K. (In Prep.). Irrelevant vibro-tactile stimuli
produce a changing-state effect: Implications for theories of
interference in short-term memory.

\textbf{Marsja}, E., Neely, G., and Ljungberg, J. K. (2018).
Investigating Deviance Distraction and the Impact of the Modality of the
To-Be-Ignored Stimuli. \emph{Exp. Psychol. 65}(2), 61-70. Scripts and
Data: \url{https://osf.io/amd2h/}

\textbf{Marsja}, E. (2017). Attention capture by sudden and unexpected
changes: a multisensory perspective (PhD thesis). Umeå University, Umeå.

\textbf{Marsja}, E., Marsh, J. E., Hansson, P., \& Neely, G. (2019).
Examining the Role of Spatial Changes in Bimodal and Uni-Modal
To-Be-Ignored Stimuli and How They Affect Short-Term Memory Processes.
\emph{Front. Psychol.}, 10.

\textbf{Marsja}, E., Neely, G., and Ljungberg, J. K. (In Prep.).
Deviance distraction in the auditory and tactile modalities after
repeated exposure: differential aspects of tactile and auditory
deviants. Scripts and Data: \url{https://github.com/marsja/study1}

\textbf{Marsja}, E., Stenbäck, V., Moradi, S., Danielsson, H., \&
Rönnberg, J. (2022). Is Having Hearing Loss Fundamentally different?
Multi-group structural equation modeling of the effect of cognitive
functioning on speech identification. \emph{Ear Hear}. Scripts:
\url{https://github.com/marsja/EANDH-D-21-00145}

Rosa, E., \textbf{Marsja}, E., \& Ljungberg, J. K. (2020). Exploring
Residual Capacity: The Effectiveness of a Vibrotactile Warning During
Increasing Levels of Mental Workload in Simulated Flight Tasks.
\emph{Aviation Psychology and Applied Human Factors, 10}(1), 13--23.

Stenbäck, V., \textbf{Marsja}, E., Hällgren, M., Lyxell, B., \& Larsby,
B. (2021). The Contribution of Age, Working Memory Capacity, and
Inhibitory Control on Speech Recognition in Noise in Young and Older
Adult Listeners. \emph{J. Speech Lang. Hear. Res., 64}(11), 4513--4523.

Stenbäck, V., \textbf{Marsja}, E., Hällgren, M., Lyxell, B., \& Larsby,
B. (Submitted). Informational masking and listening effort in
speech-recognition-in-noise -- the role of working memory capacity and
inhibitory control in older adults with and without hearing impairment.

Stenbäck, V., \textbf{Marsja}, E., Ellis, R., \& Rönnberg, J.
(Accepted). Relationships between objective and subjective outcome
measures of speech recognition in noise. \emph{Int. J. Audiol.}.

\end{document}