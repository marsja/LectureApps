\documentclass[]{article}
\usepackage{pdflscape}
\usepackage{longtable}
\usepackage{booktabs}
\usepackage{everypage}

\newcommand{\Lpagenumber}{\ifdim\textwidth=\linewidth\else\bgroup
  \dimendef\margin=0 %use \margin instead of \dimen0
  \ifodd\value{page}\margin=\oddsidemargin
  \else\margin=\evensidemargin
  \fi
  \raisebox{\dimexpr -\topmargin-\headheight-\headsep-0.5\linewidth}[0pt][0pt]{%
    \rlap{\hspace{\dimexpr \margin+\textheight+\footskip}%
    \llap{\rotatebox{90}{ Cover Letter  - Erik Marsja 
- \thepage\hspace{1pt}(\pageref{LastPage})}}}}%
\egroup\fi}
\AddEverypageHook{\Lpagenumber}%

\usepackage{adjustbox}
\usepackage{titling}
\usepackage{caption}
\usepackage{tabularx}
\usepackage{tikz}
\usepackage{xcolor}
\usepackage{fancyhdr}
\usepackage{lastpage}
\usepackage{titlesec}
\usepackage[scaled=.90]{helvet}% Helvetica, served as a model for arial

\pagestyle{fancy}
\fancyhf{}
\renewcommand{\headrulewidth}{0pt}
\renewcommand{\footrulewidth}{0.0pt}
\fancyfoot[CO,CE]{ Personligt Brev -    Erik Marsja -  \thepage\hspace{1pt}(\pageref{LastPage})}
\fancypagestyle{plain}{\pagestyle{fancy}}

\usepackage[margin=1.2in]{geometry}

% Font awesome
\usepackage{fontawesome}

\usepackage[T1]{fontenc}
\usepackage[utf8]{inputenc}

% For tables


% Tightlist

\providecommand{\tightlist}{%
  \setlength{\itemsep}{0pt}\setlength{\parskip}{0pt}}

% URLS
\usepackage[hidelinks]{hyperref}
\usepackage{breakurl}
\usepackage{float} % here for H placement parameter

% Appendix header style

\fancypagestyle{style2}{
\fancyhf{}
\fancyhead[C]{Appendix 1}
}


% For Changing the margins (Education)

\def\changemargin#1#2{\list{}{\rightmargin#2\leftmargin#1}\item[]}
\let\endchangemargin=\endlist 

%For displaying the table in landscape format
\usepackage[absolute]{textpos}

\fancypagestyle{lscape}{% 
\fancyhf{} % clear all header and footer fields 
\fancyfoot[LE]{%
\begin{textblock}{20}(1,5){\rotatebox{90}{\leftmark}}\end{textblock}
\begin{textblock}{1}(13,10.5){\rotatebox{90}{\thepage}}\end{textblock}}
\fancyfoot[LO] {%
\begin{textblock}{1}(13,10.5){\rotatebox{90}{\thepage}}\end{textblock}
\begin{textblock}{20}(1,13.25){\rotatebox{90}{\rightmark}}\end{textblock}}
\renewcommand{\headrulewidth}{0pt} 
\renewcommand{\footrulewidth}{0pt}}

\setlength{\TPHorizModule}{1cm}
\setlength{\TPVertModule}{1cm}

% Appendix
\fancypagestyle{style2}{
\fancyhf{}
\fancyhead[C]{Appendix 1}
\renewcommand{\headrulewidth}{1pt}
}



\newcommand\secbar {
    \tikz[baseline, trim left=3.2cm] 
    {
        \fill [white] (3cm,0.1ex) rectangle +(0.2cm,1.1ex);
        \draw [gray!95, fill=gray!80] (0cm,0.1ex) rectangle (3cm,1.1ex);        
    }
}
\newcommand\subsecbar {
    \tikz[baseline, trim left=0.15cm] 
    {
        \fill [white] (2cm,0.1ex) rectangle +(0.2cm,1.1ex);
        \fill [blue!40] (0cm,0.1ex) rectangle (2cm,1.1ex);      
    }
}

\newcommand\subsubsecbar {
    \tikz[baseline, trim left=0.15cm] 
    {
        \fill [white] (1cm,0.1ex) rectangle +(0.2cm,1.1ex);
        \fill [blue!40] (0cm,0.1ex) rectangle (2cm,1.1ex);      
    }
}

\titleformat{\section}{\large}{}{0cm}{\secbar}
\titleformat{\subsection}{\large}{}{0cm}{\normalfont\sffamily\Large\bfseries\subsecbar}
\titleformat{\subsubsection}{}{}{0cm}{\normalfont\sffamily\large\bfseries}

% No first line paragraph indent
\usepackage{parskip}
\usepackage{enumitem}


\titlespacing\section{0pt}{12pt plus 4pt minus 2pt}{4pt plus 2pt minus 2pt}
\titlespacing\subsection{0pt}{12pt plus 4pt minus 2pt}{4pt plus 2pt minus 2pt}
\titlespacing\subsubsection{0pt}{12pt plus 4pt minus 2pt}{4pt plus 2pt minus 2pt}

\newcolumntype{Y}{>{\centering\arraybackslash}X}

\begin{document}

\centerline{\huge \textbf{Erik Marsja} | \textcolor{darkgray}{Cover Letter}}

\vspace{2 mm}

\hrule

\begin{table}[h]
\centering
\begin{tabularx}{\textwidth}{@{}lYl@{}}
\textbf{Home Address}: & & 
\\Tvistevägen 26, SE-907 36 Umeå, Sweden & & 
\\\\

 \faPhone \hspace{1 mm}  +4670-3633662  \hspace{1 mm}  &  & \faEnvelopeO \hspace{1 mm} \href{mailto:}{\tt \href{mailto:erik.marsja@liu.se}{\nolinkurl{erik.marsja@liu.se}}} \hspace{1 mm}  \\
 \faGlobe \hspace{1 mm} \href{http://www.marsja.se}{\tt www.marsja.se}   &  & \faGithub \hspace{1 mm} \href{http://github.com/marsja}{\tt marsja} \hspace{1 mm}  \\
 \multicolumn{3}{c}{}
 \\\hline
\end{tabularx}
\end{table}

Dear Hiring Committee,

I want to express my interest in the position as a lecturer in cognitive
science, at the Department of Computer Science and Information,
Linköping University. Currently, I am working as a post-doc researcher
at the Division of Disability Research, Department of Behavioural
Sciences and Learning. I find it very interesting to work with teaching,
collaborate, and carry out research in technological areas (e.g.,
computer science). Mainly because it is an excellent opportunity to make
use of some of the many skills I developed during my time as a graduate
student and post-doc, including my strong skills in research, statistics
and methods, teaching, and programming. Overall, my research has been
focused on cross-modal aspects of attention and distraction, hearing,
age, and cognition. I have acquired good knowledge in the field of
multisensory attention and cognitive hearing science.

Additionally, I also have a strong interest in programming and data
analysis. These skills have been put to use in collaboration with
national and international researchers. During my Ph.D.~studies, I have
used specially designed vibration equipment, but also created my own
vibration equipment using open source products (e.g., Arduino) and
programming. I have also used programming to analyze data (e.g., R and
Python) using e.g.~machine learning and/or multivariate statistics.

My teaching responsibilities include both quantitative and qualitative
research methodology, applied cognitive psychology, and cognitive
psychology. I have been involved in the supervision of projects where
students, together with external companies, perform projects where they
apply basic research knowledge. These projects, including the thesis' I
have supervised, cover areas such as psychology, human-computer
interaction (e.g., evaluating the design of products), machine learning,
and collecting and analyzing data for companies. In addition, I have
participated in several international conferences, taught for
international students, and collaborated with international researchers.

I have a background in cognitive science (an MSc. degree) and, in
addition to my expertise in cognitive psychology, thus have a broad
knowledge in all cognitive science subjects, including computer science
(e.g., programming, human-computer interaction, and artificial
intelligence).

During my spare time, I enjoy reading, cooking, and baking sourdough
bread. I also enjoy hiking, and spending time with my family and
friends. Fishing is a relatively new hobby. Occasionally, I enjoy
blogging (i.e., write programming tutorials).

I look forward to growing in an intellectually challenging environment
and to work with a technical focus, expand my skills to human-computer
interaction and machine learning, and research - both in collaboration
with industry and basic research. I strongly believe that my experience
and merits are of great benefit to the department as well as the
research community.

Do not hesitate to contact me via +46703633662 or
\href{mailto:erik.marsja@liu.se}{\nolinkurl{erik.marsja@liu.se}}. Thank
you for your consideration and your time.

Kind Regards, Erik Marsja

\end{document}