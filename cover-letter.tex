\documentclass[]{article}
\usepackage{pdflscape}
\usepackage{longtable}
\usepackage{booktabs}
\usepackage{everypage}

\newcommand{\Lpagenumber}{\ifdim\textwidth=\linewidth\else\bgroup
  \dimendef\margin=0 %use \margin instead of \dimen0
  \ifodd\value{page}\margin=\oddsidemargin
  \else\margin=\evensidemargin
  \fi
  \raisebox{\dimexpr -\topmargin-\headheight-\headsep-0.5\linewidth}[0pt][0pt]{%
    \rlap{\hspace{\dimexpr \margin+\textheight+\footskip}%
    \llap{\rotatebox{90}{ Cover Letter  - Erik Marsja 
- \thepage\hspace{1pt}(\pageref{LastPage})}}}}%
\egroup\fi}
\AddEverypageHook{\Lpagenumber}%

\usepackage{adjustbox}
\usepackage{titling}
\usepackage{caption}
\usepackage{tabularx}
\usepackage{tikz}
\usepackage{xcolor}
\usepackage{fancyhdr}
\usepackage{lastpage}
\usepackage{titlesec}
\usepackage[scaled=.80]{helvet}% Helvetica, served as a model for arial

\pagestyle{fancy}
\fancyhf{}
\renewcommand{\headrulewidth}{0pt}
\renewcommand{\footrulewidth}{0.0pt}
\fancyfoot[CO,CE]{ Personligt Brev -    Erik Marsja -  \thepage\hspace{1pt}(\pageref{LastPage})}
\fancypagestyle{plain}{\pagestyle{fancy}}

\usepackage[margin=1in]{geometry}

% Font awesome
\usepackage{fontawesome}

\usepackage[T1]{fontenc}
\usepackage[utf8]{inputenc}

% For tables


% Tightlist

\providecommand{\tightlist}{%
  \setlength{\itemsep}{0pt}\setlength{\parskip}{0pt}}

% URLS
\usepackage[hidelinks]{hyperref}
\usepackage{breakurl}
\usepackage{float} % here for H placement parameter

% Appendix header style

\fancypagestyle{style2}{
\fancyhf{}
\fancyhead[C]{Appendix 1}
}


% For Changing the margins (Education)

\def\changemargin#1#2{\list{}{\rightmargin#2\leftmargin#1}\item[]}
\let\endchangemargin=\endlist 

%For displaying the table in landscape format
\usepackage[absolute]{textpos}

\fancypagestyle{lscape}{% 
\fancyhf{} % clear all header and footer fields 
\fancyfoot[LE]{%
\begin{textblock}{20}(1,5){\rotatebox{90}{\leftmark}}\end{textblock}
\begin{textblock}{1}(13,10.5){\rotatebox{90}{\thepage}}\end{textblock}}
\fancyfoot[LO] {%
\begin{textblock}{1}(13,10.5){\rotatebox{90}{\thepage}}\end{textblock}
\begin{textblock}{20}(1,13.25){\rotatebox{90}{\rightmark}}\end{textblock}}
\renewcommand{\headrulewidth}{0pt} 
\renewcommand{\footrulewidth}{0pt}}

\setlength{\TPHorizModule}{1cm}
\setlength{\TPVertModule}{1cm}

% Appendix
\fancypagestyle{style2}{
\fancyhf{}
\fancyhead[C]{Appendix 1}
\renewcommand{\headrulewidth}{1pt}
}



\newcommand\secbar {
    \tikz[baseline, trim left=3.2cm] 
    {
        \fill [white] (3cm,0.1ex) rectangle +(0.2cm,1.1ex);
        \draw [gray!95, fill=gray!80] (0cm,0.1ex) rectangle (3cm,1.1ex);        
    }
}
\newcommand\subsecbar {
    \tikz[baseline, trim left=0.15cm] 
    {
        \fill [white] (2cm,0.1ex) rectangle +(0.2cm,1.1ex);
        \fill [blue!40] (0cm,0.1ex) rectangle (2cm,1.1ex);      
    }
}

\newcommand\subsubsecbar {
    \tikz[baseline, trim left=0.15cm] 
    {
        \fill [white] (1cm,0.1ex) rectangle +(0.2cm,1.1ex);
        \fill [blue!40] (0cm,0.1ex) rectangle (2cm,1.1ex);      
    }
}

\titleformat{\section}{\large}{}{0cm}{\secbar}
\titleformat{\subsection}{\large}{}{0cm}{\normalfont\sffamily\Large\bfseries\subsecbar}
\titleformat{\subsubsection}{}{}{0cm}{\normalfont\sffamily\large\bfseries}

% No first line paragraph indent
\usepackage{parskip}
\usepackage{enumitem}

\setlength{\parskip}{0cm}


\titlespacing\section{0pt}{12pt plus 4pt minus 2pt}{1pt plus 1pt minus 2pt}
\titlespacing\subsection{0pt}{12pt plus 4pt minus 2pt}{1pt plus 1pt minus 2pt}
\titlespacing\subsubsection{0pt}{12pt plus 4pt minus 2pt}{4pt plus 2pt minus 2pt}

\newcolumntype{Y}{>{\centering\arraybackslash}X}

\begin{document}

\centerline{\huge \textbf{Erik Marsja} | \textcolor{darkgray}{Personligt
Brev}}

\vspace{1mm}

\hrule

\begin{table}[h]
\centering
\begin{tabularx}{\textwidth}{@{}lYl@{}}
\textbf{Home Address}: & & 
\\Tvistevägen 26, SE-907 36 Umeå, Sweden & & 
\\\\

 \faPhone \hspace{1 mm}  +4670-3633662  \hspace{1 mm}  &  & \faEnvelopeO \hspace{1 mm} \href{mailto:}{\tt \href{mailto:erik.marsja@liu.se}{\nolinkurl{erik.marsja@liu.se}}} \hspace{1 mm}  \\
 \faGlobe \hspace{1 mm} \href{http://www.marsja.se}{\tt www.marsja.se}   &  & \faGithub \hspace{1 mm} \href{http://github.com/marsja}{\tt marsja} \hspace{1 mm}  \\
 \multicolumn{3}{c}{}
 \\\hline
\end{tabularx}
\end{table}

Till den det berör,

\hfill\break
Jag vill uttrycka mitt intresse för positionen som biträdande lektor i
funktionsnedsättning och samhälle vid institutionen för
beteendevetenskap och lärande, Linköpings universitet. För tillfället
arbetar jag som postdoktor vid avdelningen för funktionsnedsättning och
samhälle, institutionen för beteendevetenskap och lärande, Linköpings
Universitet och som forskare vid människan i transportsystemet, Statens
Väg- och Transportforskningsinstitut (VTI). Att forska och undervisa
inom funktionsnedsättning och samhälle finner jag väldigt intressant och
stimulerande. Detta framför allt eftersom jag det ger mig ett tillfälle
att tillämpa de färdigheter jag anförskaffat mig som doktorand och
postdoktor, inklusive mina starka färdigheter i forskning, avancerad
statistik, och metod. Sedan min forskarutbildning har jag brunnit för
öppen vetenskap (var exv. med vid grundadet av Open Science Community
Sweden) och har som målsättning att för-registrera mina egna projekt,
laddar upp analysskript och data för mina studier när det är möjligt,
och strävar således efter både öppen- såväl som reproducerbarhet.
Generellt sett har min forskning fokuserat på multisensorisk perception
och uppmärksamhet, kognitiva och psykosociala aspekter av
hörselnedsättning. Det vill säga, min forskning tillika planerad
forskning innefattar ett spann av metoder inom funktionsnedsättning och
samhälle, psykologi samt kognitionsvetenskap. Forskningsmässigt strävar
jag efter att bidra till både teori i form av grundforskning men även
att i andra projekt kunna bidra mer direkt med tillämpad kunskap.

\hfill\break
Mina undervisningserfarenheter inkluderar både kvantitativ och
kvalitativ forskningsmetod, tillämpad kognitiv psykologi,
kognitionspsykologi, neurovetenskap, och handikappvetenskap (numer
benämnt som funktionsnedsättning och samhälle). Jag har varit involverad
i handledning av projekt där studenter, tillsammans med företag, utför
projekt där de ska tillämpa grundforskning (exv. arbetat utifrån den
biopsykosociala modellen). Dessa projekt, inklusive uppsatshandledning,
har innefattat flera områden inom psykologi, människo-datorinteraktion,
kognitions- och handikappvetenskap. Vidare har jag deltagit i flera
internationella konferenser, undervisat för internationella studenter,
och samarbetat med internationella forskare.

På min fritid tycker jag om att läsa, laga mat, och baka surdegsbröd.
Jag tycker även om att fjällvandra, jogga, odla och spendera tid med min
familj och vänner. De senaste åren har jag även fattat tycke för att
fiska. Ibland så skriver jag på min Python och R-programmeringsblogg. Är
något av en kaffenörd.

\hfill\break
Jag ser fram emot att växa i en intellektuellt utmanande miljö och att
arbeta med min forskning tillika med undervisning inom
funktionsnedsättning och samhälle, psykologi och kognitionsvetenskap.
Slutligen, är jag övertygad om att min erfarenhet och mina meriter
skulle vara av stort gagn för både avdelningen för funktionsnedsättning
och samhälle och Linköpings universitet.

\hfill\break
Tveka inte att ta kontakt med mig via 0703633662 eller
\href{mailto:erik.marsja@liu.se}{\nolinkurl{erik.marsja@liu.se}}. Tack
för ert övervägande och er tid.

\hfill\break
Med Vänlig Hälsning,

\hfill\break
Dr Erik Marsja

\end{document}